\chapter{Introduction}

\section{Motivation}

Suppose a famous auction house like Sotheby's\footnote{\url{https://www.sothebys.com/en}} wishes to maximize its profits selling a work of art. Which auction format should it use? Auctions can be \textit{open-bid} or \textit{sealed-bid}, depending on whether bidders must reveal their bids publicly. An auction can also be \textit{single-round} or \textit{multi-round}, where bidders are allowed to bid multiple times, as in the case of English\footnote{\url{https://en.wikipedia.org/wiki/English_auction}} or Dutch\footnote{\url{https://en.wikipedia.org/wiki/Dutch_auction}} auctions. Multiple goods may be sold, sometimes packaged or \textit{bundled} together for a discount. \textit{Reserve prices} for these goods can be posted so that a seller will only sell if bids are above them. In \textit{all-pay} auctions, all bidders are required to pay irrespective of whether they submit the winning bid. Additionally, auctions can even require bidders pay an amount different than their bid, as in the case of a \textit{second-price} auction\footnote{Most commonly contrasted with a \textit{first-price} auction, where the bidder with the highest bid wins the good(s) and pays the value of their bid.}, where the bidder with the highest bid pays wins the good(s) but only pays the value of the second highest bid.

Perhaps surprisingly, it turns out this question has an associated deep body of economic theory, stretching back to Nobel Laureate William Vickrey's \autocite*{vickrey1961} seminal research demonstrating the revenue equivalence of first and second-price auctions. Auction theory---more generally, the theory of \textit{mechanism design}---has illuminated many facets of the problem of designing auctions. Theoretical work has extended Vickrey's famous result, characterizing the entire class of profit-maximizing auctions for selling a single good \autocite{myerson1981optimal}. We know that if multiple goods are sold offering selling them together \autocite{adams1976} or offering a lottery, where the bidder is unsure which good they end up with in the event they win \autocite{thanassoulis2004}, can increase profits. Theoretical work has even extended our understanding of `rational' behavior in auctions \autocite{li2017}, adding to our scientific arsenal of concepts that we can use to characterize how people will respond to different institutional designs. These are only a few examples from the decades of research in theoretical economics that applies to designing auctions.

This theory also applies much more widely than one might initially guess. An auction can be represented mathematically by a pair of functions\footnote{A comprehensive model of multidimensional auction design relevant to the microeconomic portion of this thesis is introduced in section \ref{sec_model}.}. An \textit{allocation function} maps bids to outcomes. A \textit{transfer function} maps bids to payments. With this pair of functions it is possible to characterize the wide range of variations in auction format introduced above. Moreover, this mathematical representation covers a surprisingly wide range of design problems. Parallels can immediately be drawn to the problem of \textit{price screening} faced by a monopolist\footnote{The problem of price screening is commonly known as \textit{second-degree price discrimination}, where a seller offers a range of options to buyers in order to reveal buyers' private information concerning the value of the seller's goods.}. Any bilateral trading platform (e.g., a stock market or exchange) can be characterized as a \textit{double auction} with multiple buyers and sellers, offering an alternative perspective on the design of markets. The theory of auction design has even been applied to the problem of data acquisition and survey sampling \autocite{roth2012surveys}, an example covered in more detail in chapter \ref{ch_reflexivity}.

What is the scientific value of this body of economic theory? Is it actually useful for designing institutions and policy-making? Economists and philosophers of science are divided on this question. Well-known, award-winning microeconomic theorists like Ariel Rubinstein believe economic theory merely amounts to a collection of fables \autocite{rubinstein2012} whereas other equally well-known, award-winning microeconomic theorists assert that this kind of scientific theory has predictive value and sharpens the intuitions of economists working on these problems \autocite{roth2019}. Philosophers of science are no less split, with some arguing that economic theory played an incredibly limited role in some of the major successes of auction design in the 1990s \autocite{nikkhah2008} and others asserting this kind of theory is true in a strong, ontological sense \autocite{ross2008}.

This thesis represents a philosopher of science's attempt to better understand a body of scientific theory combined with an attempt to contribute to it. The overarching goal is to illuminate the value and uses of economic theory for questions of design. The scientific portion of this thesis contributes to our theoretical understanding of auction design. The philosophical orientation of this thesis is prescriptive. The philosophical conclusions of this thesis directly entail recommendations regarding how to improve science. Thus, the novel philosophical contributions of this thesis are intended to be relevant for economists and social scientists interested in the foundational problems that concern the study of human behavior.


% \section{Contributions


\section{(A Very Brief) Introduction to Design Economics}\label{intro_sec_whatis}

Design economics is both a collection of institutional design problems as well as an approach to doing economics\footnote{See section \ref{fable_sec_design} for an extended introduction to this topic, along with consideration of two major empirical successes from the 1990s: the Federal Communication Commision's (FCC) 1994 radio spectrum auction and the National Resident Matching Program's (NRMP) redesign of the matching algorithm in 1996.}. Economist Al Roth has called design economics ``the part of economics intended to further the design and maintenance of markets and other economic institutions'' \autocite[1341]{roth2002}. It is often used synonymously with `market design'. At its core, the theoretical content of design economics models the choice a principal (designer) makers concerning how to design an institution to achieve some goal. Common goals include profit maximization (optimality), welfare maximization, efficiency, stability, and equity. Oftentimes, this work is applied in orientation. Economists increasingly consult on questions of policy and empirical and computational work play a prominent role alongside economic theory (see, for first-hand accounts of this phenomenon, \cite{binmore2002, roth2019, sönmez2023minimalist}). 

What are some examples of institutional design problems tackled by economists working in design economics? Auction design, introduced above, is a core topic in design economics. Both empirical and theoretical work in economics tries to understand the trade-offs that come with choosing one type of auction format over another. Matching problems are another major topic. These concern problems like matching students to schools, new doctors to hospitals, and organs to transplant recipients. The flexibility of economic theory in this domain means that it can be applied well beyond the setting that motivated its development. As mentioned above, auction theory also applies to problems of price discrimination and data acquisition, a topic that was pioneered by computer scientists (see, for example, \cite{roth2012surveys, cai2015}). Although economists aided policymakers on questions of institutional design in the 1990s, since then computer scientists have been increasingly involved in applying economic theory to problems in and adjacent to computer science\footnote{Computer scientists have also been at the forefront of studying computational aspects of economic theory (see, for example, \cite{nisan2007algorithmic}).}. Although the canonical problems of auction design and matching markets were responsible for establishing the domain of design economics, its successes have broadened its scope of application and attracted scientists from neighboring disciplines.

The intellectual origins of design economics can be traced to the 1960s and 70s. Early work on auctions \autocite{vickrey1961} and matching mechanisms \autocite{gale1962} blossomed into substantial theoretical literature in mechanism design and matching theory. Mechanism design, in the spirit of Hurwicz \autocite*{hurwicz1972}, became a discipline in its own right, which now boasts its own specialized journal, the \textit{Review of Economic Design} (created in 1994). A similar story exists for the development of matching theory. Ultimately, both of these disciplines blossomed alongside related empirically oriented literature on the outcomes of particular forms of institutional design\footnote{See, for example, discussions by Roth \autocite*{roth2002} and Myerson \autocite*{myerson2008}.}. To understand the theoretical orientation of this type of scientific modeling, one of the founders of mechanism design, Leonid Hurwicz, has described it as follows:
\begin{quote}
	in a design problem, the goal function is the main given, while the mechanism is the unknown. Therefore, the design problem is the inverse of traditional economic theory, which is typically devoted to the analysis of the performance of a given mechanism. \autocite[30]{hurwicz2006designing}
\end{quote}
\noindent Though economic theories of design economics do examine the performance of particular institutions, much theoretical work takes place in the same vein that Hurwicz outlined. As I will show in chapter \ref{ch_fables}, these approaches are complementary.

As noted above, in design economics empirical and computational work are seen as ``natural complements'' \autocite[1363]{roth2002} to theory. This has led to what some economists have called the `engineering approach' to questions of institutional design. Al Roth has elaborated on this approach by way of analogy with the construction of suspension bridges:
\begin{quote}
	The simple theoretical model in which the only force is gravity, and beams are perfectly rigid, is elegant and general. But bridge design also concerns metallurgy and soil mechanics, and the sideways forces of water and wind. Many questions concerning these complications can’t be answered analytically but must be explored using physical or computational models. These complications, and how they interact with the parts of the physics captured by the simple model, are the domain of the engineering literature. \autocite[1342]{roth2002}
\end{quote}
\noindent The key idea behind the engineering approach is that computation and experimentation ``fill[] the gaps between theory and design'' \autocite[p1374]{roth2002}. Computational methods analyze settings that are too complex to solve analytically and laboratory experiments offer predictions about how people will behave in these environments. This kind of approach motivates the use of simulations in chapter \ref{ch_auctions} of this thesis.

The rise of design economics has been tied to a change in perspective in how economists view their subject matter. This change has even recently been lamented by economic theorists:
\begin{quote}
    If I had to name one major shift in the sensibilities of economic theorists in the past half century, a prime candidate would be the way we conceptualize markets---from quasi-natural phenomena admired from afar to man-made institutions whose design can be tweaked by economist-engineers. \autocite[p137]{spiegler2024}
\end{quote}
\noindent I believe this change reflects progress in the science of economics. The turn away from contemplating man-made institutions as ``quasi-natural phenomena'' characterizes a step in the evolution of the `dismal science' of economics towards a better understanding of our (socially constructed) world. I hope to convince the reader that not only is the theoretical contents of this emerging scientific domain philosophically interesting in its own right but also that it can be fruitfully applied to address foundational problems in other areas of social science.


\section{Structure of the Thesis}

This thesis consists of three chapters: two in the philosophy of science, one focusing on economic theory in design economics (chapter \ref{ch_fables}) and the other on statistics in social science (chapter \ref{ch_reflexivity}), as well as an additional chapter in auction theory (chapter \ref{ch_auctions}). The chapters in this thesis can be seen as answering the following related questions:

\begin{itemize}
	\item \textit{`What is the scientific value of economic theory in design economics?'} (chapter \ref{ch_fables})

	\item \textit{`What is an example of a theoretical contribution in design economics?'} (chapter \ref{ch_auctions})

	\item \textit{`What other problems might this body of theory help with?'} (chapter \ref{ch_reflexivity})
\end{itemize}

\noindent Although each chapter is self-contained, the two chapters in philosophy of science are ordered in the following way: the conclusion of chapter \ref{ch_fables} on the value of economic theory in design economics---that economic theory explains the successes of design economics---strengthens the case for its application to problems of reflexive measurement in chapter \ref{ch_reflexivity}. The recommendation to apply the theory of mechanism design to the problem of designing incentive-compatible measurements (\ref{sec_reflex_mechdes}) follows from the conclusion concerning the value of this theory.

Chapter \ref{ch_fables} offers a \textit{minimal non-fable} account of economic theory in design economics. The goal of this chapter is to offer a philosophical refutation of the position that economic theory does not ``produce conclusions of real value'' \autocite[37]{rubinstein2012}. My argument draws inspiration from a classic philosophical argument that argues for the truth of scientific theories in virtue of their success---the \textit{no-miracles} argument for scientific realism \autocite{putnam1975}. However, this chapter adapts arguments for realism to scientific domains where notions of truth are a ``non-starter'' \autocite[328]{alexandrova2009}. Instead of arguing for the strong conclusion concerning the truth of economic theory in design economics, I argue for a much weaker thesis: economic theory explains the successes of design economics. I develop an account of the economic theory of design which I believe best accounts for the recent empirical successes of economics\footnote{These successes are outlined in more detail in section \ref{fable_sec_design} and a brief overview of design economics is provided below (section \ref{intro_sec_whatis}).}. Unlike existing philosophical accounts of the role of economic theory in design economics (see, for example, \cite{alexandrova2009, ross2008}) the argument developed here is sensitive to contemporary contributions by economists and computer scientists as well as captures the subtle differences in the kinds of theory used to inform questions of institutional design.

% The scientific contributions of this thesis concern theoretical questions of optimal auction design in the multidimensional setting of a single good with multiple quality levels. These are covered in chapter \ref{ch_auctions}. These contributions fall squarely in the tradition of design economics, which views computation and experiment as ``natural complements'' \autocite[1342]{roth2002} to theory. The contributions are generative in that they ``suggest[] an agenda for theoretical work'' \autocite[1363]{roth2002}. Using simulations, the optimality of the \textit{exclusive-buyer mechanism}\footnote{The exclusive-buyer mechanism is formally defined in section \ref{subsec_ebm}. For a review of related work, see the literature review in section \ref{sec_litreview}.} is explored alongside other conjectures concerning qualitative characteristics of profit-maximizing mechanisms in this multidimensional setting. Surprisingly, I find the exclusive-buyer mechanism is optimal in settings where randomization is not required for profit maximization. Additionally, I find that the \textit{exclusion region}---the measure of the type space which does not receive the good in equilibrium---does not change with the number of bidders and is sometimes measure zero.

The scientific contributions of this thesis can be found in chapter \ref{ch_auctions}. These concern developing an improved understanding of optimal auction design in the multidimensional setting of a single good with multiple quality levels. Optimal multidimensional auction design problems are notoriously difficult to solve analytically and for this reason, the setting is explored using simulations. These contributions fall squarely in the tradition of design economics, which views computation and experiment as ``natural complements'' \autocite[1342]{roth2002} to theory. Using simulations, the optimality of the \textit{exclusive-buyer mechanism}\footnote{The exclusive-buyer mechanism is formally defined in section \ref{subsec_ebm}. For a review of related work, see the literature review in section \ref{sec_litreview}.} is explored alongside other conjectures concerning qualitative characteristics of profit-maximizing mechanisms in this multidimensional setting. Surprisingly, I find the exclusive-buyer mechanism is optimal in settings where randomization is not required for profit maximization. Additionally, I find that the \textit{exclusion region}---the measure of the type space which does not receive the good in equilibrium---does not change with the number of bidders and is sometimes measure zero. The contributions are generative in that they ``suggest[] an agenda for theoretical work'' \autocite[1363]{roth2002} and can be used to guide future theoretical research on optimal multidimensional auction design.

The final substantive chapter of the thesis advances a general philosophical account of observer effects in the social sciences (chapter \ref{ch_reflexivity}). My account unifies almost a century of sustained research on this topic by both scientists and philosophers. In my view, observer effects can be understood as problems of \textit{reflexive measurement}, which occur when people are aware of their status as objects of scientific investigation. Viewed at a sufficient level of philosophical abstraction, this characterization not only recovers well-known problems in the social sciences like `Goodhart's Law' and the \textit{Hawthorne effect} but also sheds new light on how to overcome them. A typical understanding of measurement error will fail to account for the distribution shift caused by a reflexive measurement. The conclusions of chapter \ref{ch_fables} facilitate a connection to recent developments in theoretical computer science in the field of \textit{incentive compatible learning}. I argue that the economic theory of design economics can address problems of reflexive measurement where other ascientific, ``purely statistical'' \autocite[319]{marget1929} approaches fail to do so. 

% The conclusion of chapter \ref{ch_fables} concerning the value of economic theory in the context of design is directly applicable in chapter \ref{ch_reflexivity}, which concerns developing a general philosophical account of observer effects in social science. The problem of \textit{reflexive measurement} characterizes any science where the agents who are under investigation are aware of their status as a target of scientific inquiry. This problem has been widely studied for almost a century, drawing attention from noted economists \autocite{morgenstern1928}, sociologists \autocite{merton1948}, and philosophers of science \autocite{popper1953}. This chapter explicitly connects this foundational problem in social science with contemporary applied statistical practice as well as recent developments in theoretical computer science in the field of \textit{incentive compatible learning}. I argue that the economic theory of design economics can address problems of reflexive measurement where other ascientific, ``purely statistical'' \autocite[319]{marget1929} approaches fail to do so. 



% ---- old `structure of thesis section



% Chapter \ref{ch_fables} directly addresses Rubinstein's \autocite*{rubinstein2006, rubinstein2012} contention concerning the role of economic theory as a collection of fables. The chapter puts forward a \textit{minimimal non-fable} account of economic theory for the domain of design economics. The scope of this chapter is deliberate. Insofar as design economics boasts a number of impressive achievements, these success stories render questions concerning the usefulness of economic theory increasingly urgent. Does economic theory explain these successes? I believe so. By adapting a \textit{no-miracles} argument \autocite{putnam1975} for scientific realism, I argue that it is unlikely (miraculous) that the theory common to these success stories does not, in part, explain their success. Rubinstein's \textit{theory-as-fable} account is found untenable. The particular \textit{projective} quality \autocite{guala2001} of economic theory in this domain is explored to illuminate how economic theory might serve to bring about these successes

% With a philosophical argument for the scientific value of economic theory in design economics firmly in place, chapter \ref{ch_auctions} tackles the question of optimal multidimensional auction design in the setting of a single good with multiple quality levels. In the spirit of design economics explored in the previous section, simulations are used to explore this analytically intractable setting. These simulations are derived from modifying the plane-cutting algorithm developed by Belloni et al. \autocite*{belloni2010multidimensional} to improve performance and facilitate exploration of a wide variety of settings. In particular, I explore the performance of the \textit{exclusive-buyer mechanism}. This is a deterministic mechanism that allocates the good according to a multidimensional analog of the single-dimensional virtual value (see section \ref{subsec_ebm} for details). I find the simple exclusive-buyer mechanism not only well approximates the revenue from the optimal mechanism but the interim allocations are qualitatively similar to those of the optimal mechanism when the optimal mechanism is deterministic. (Surprisingly, deterministic mechanisms are often optimal in multidimensional settings). Additionally, two conjectures are investigated concerning the measure of types excluded by the optimal mechanism in equilibrium. 

% The final chapter \ref{ch_reflexivity} offers a unified philosophical account of observer effects in social science. When agents are aware their behavior is subject to scientific inquiry they often act in ways that render measurements of them unreliable. This is the problem of \textit{reflexive measurement}, the central topic of this chapter. In order to develop this novel account, I provide a general characterization of reflexivity which encompasses the full scope of scientific practice: theorizing, prediction, measurement, etc. The characterization captures the insights of contemporary philosophers of science working on reflexive prediction alongside observations by scientists grappling with observer-type effects in the course of their research.






% ------ first attempt

% \section{Motivation}

% Two fundamental questions animate this thesis. First, `\textit{How should a philosopher's scientific investigations constrain their philosophy?}' Second, `\textit{How should a scientist's philosophical convictions inform the way they practice science?}'. Although neither of these questions are addressed directly, they have indirectly motivated the entirety of this thesis. Thus, this work reflects a philosopher's extended scientific engagement with microeconomic theory, particularly auction design, and a philosophical attempt to understand the value of this theory.

% Like Marx, my philosophical inclinations towards science are unwaveringly prescriptive. Insofar as a `better' science can be identified and argued for, the path to get there is part of the conclusion of the argument. This is reflected in both chapters on the philosophy of science and mechanism design. In Chapter \ref{ch_fables}, I propose an account of economic theory in the discipline of `design economics' \autocite{roth2002} that recovers the prescriptive elements of design---what philosopher of science Francesco Guala \autocite*{guala2001} has called \textit{projective} theory. In my philosophical view, the attractive scientific quality of this kind of economic theory is its constructive orientation toward designing a better world. Arriving at this conclusion, however, was entirely incidental. Without my work in multidimensional auction design, I would never have understood the significance of the perspective of design. 

% Seeing the social world through the eyes of economic design was a revelation for me. This nascent sensibility has been (unfortunately, in my opinion) lamented by economic theorist Ran Spiegler \autocite*[p137]{spiegler2024}, who sagely writes:
% \begin{quote}
%     If I had to name one major shift in the sensibilities of economic theorists in the past half century, a prime candidate would be the way we conceptualize markets---from quasi-natural phenomena admired from afar to manmade institutions whose design can be tweaked by economist-engineers. 
% \end{quote}
% \noindent This ``shift in sensibility'' opened my eyes to how mechanism design could be applied to problems I formerly encountered as a data scientist working in political analytics. The techniques of \textit{incentive compatible learning} represent, in my eyes, a radical new scientific approach to tackling the problem of reflexivity or performativity---what social scientists would call instances of the \textit{Hawthorne Effect} (see \cite{landsberger1958}). This is covered in chapter \ref{ch_reflexivity}. Here, my work on auction design was directly responsible for changing my convictions about how to overcome the scientific problem of misaligned incentives in measurement.

% This thesis is also partly a scientific enterprise. The work on optimal multidimensional auction design in chapter \ref{ch_auctions} falls very much in the vein of `design economics' which is also a target of philosophical investigation. Building on the work of Belloni et al. \autocite*{belloni2010multidimensional}, I developed a novel algorithm to explore the qualitative features of optimal mechanisms in multidimensional settings that are analytically intractable. This project is informed by the philosophical attitude towards simulation which views it as a ``natural complement'' \autocite[1342]{roth2002} to theory. This work is generative. Its aim, much in the vein of other work in design economics, is to better understand a complex theoretical setting with an eye to uncovering simple mechanisms, unexpected features, and ``suggest an agenda for [future] theoretical work'' \autocite[1363]{roth2002}.

% Contrary to the claims of Ariel Rubinstein, I disagree economic theory amounts to a mere `collection of fables' \autocite{rubinstein2006,rubinstein2012}. Again, like Marx, I find such a conviction entirely wrongheaded. (So much so I struggle to contain my incendiary writing). To decry the role of theory as fable is to remove our ability to evaluate it in virtue of its ability to materially improve our lives. And insofar as the profession of economics continues to champion the success of theory Rubinstein's contention places it beyond reproach---\textit{all} theory is fable, there is no `better' or `worse'. I believe such a position can only stymie progress in economics. I hope to convince the reader that economic theory, in particular mechanism design, can be used to design a better world. Moreover, the basis for this conviction is rooted not in ethics or political philosophy but instead in an understanding of philosophy of science.




% \section{(A Very Brief) Introduction to Mechanism Design}

% Before outlining the structure of this thesis in more detail it is helpful to briefly describe the topic of mechanism design. At a minimum, philosophers of science ought to better understand the body theory that is the target of philosophical investigation. The particular context of multidimensional auction theory relevant to the scientific portion of this thesis is introduced in much greater detail in section \ref{sec_model}. 

% \textit{Mechanism design} is a type of game theory that deals with games of incomplete information. It is often lumped together with the field of \textit{adverse selection}, which studies how a principal interacts with an agent with unknown characteristics (see, for details, \cite{salanie2005economics}). Canonical examples of adverse selection problems are those of insurance policy: how much should I charge you for health insurance if I am unsure about your smoking habits? Mechanism design, however, is more generally concerned with designing games that implement a social choice function. A \textit{mechanism} is ``essentially a set of rules to govern the interactions of the parties'' \autocite[21]{milgrom2004}. One of the founders of mechanism design, Leonid Hurwicz, has described the type of scientific modeling as follows: 
% \begin{quote}
% 	in a design problem, the goal function is the main given, while the mechanism is the unknown. Therefore, the design problem is the inverse of traditional economic theory, which is typically devoted to the analysis of the performance of a given mechanism. \autocite[30]{hurwicz2006designing}
% \end{quote}
% \noindent Most commonly, though by no means exclusively, mechanism design is concerned with designing economic institutions like auctions, markets, and other trading platforms\footnote{See Al Roth's \autocite*{roth2002} introduction to design economics for an overview of common applications of mechanism design.}. However, mechanism design extends to domains in political science and even to problems in statistics, as are considered in more detail in chapter \ref{ch_reflexivity}.

% Crucially, as a body of scientific theory,
% \begin{quote}
% 	[t]he distinguishing characteristic of the mechanism-design approach is that the principal is assumed to choose the mechanism that maximizes her expected utility, as opposed to using a particular mechanism for historical or institutional reasons. \autocite[243]{fudenberg1991game}
% \end{quote}
% \noindent It is important to note that ``maximizing expected utility'' is sufficiently neutral with respect to the objective function so as to encompass problems like revenue (profit) maximization for the principal and also welfare maximization, where all players' utilities are maximized. A direct corollary of this is that the theory can be interpreted both normatively and descriptively. Furthermore, interpretations of mechanism design models are blurred according to how literally one takes the model \autocite{bergemann2019}. Like other forms of theoretical economic analysis, equilibrium notions are invoked (called \textit{solution concepts}) which ``predict the outcome and then evaluate[] the outcome according to the objective'' \autocite[21]{milgrom2004}.

% There is no single approach to tackling problems of mechanism design. Economists differ (often bitterly) on the appropriate approach to problems of institutional design. In its canonical form, the aim of mechanism design is to identify the mechanism that maximizes performance according to some specified objective (i.e., maximizing the principal's expected utility). Alternative approaches exist. These include the `Wilson Doctrine', ``which holds that practical mechanisms should be simple and designed without assuming the designer has very precise knowledge about the economic environment in which the mechanism will operate'' \autocite[23]{milgrom2004}. The `Minimalist Market Design' approach also uses game theory but assumes the designer is more constrained in their ability to implement mechanisms that maximize the objective function (see, for discussion, \cite{sönmez2023minimalist}).




