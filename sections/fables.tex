\chapter{A Realist Argument for Design Economics}\label{ch_fables}

\section{Introduction}

This chapter advances a \textit{minimal non-fable} account of economic theory and its role in the emerging domain of `design economics' \autocite{roth2018}. Though closely related to arguments concerning philosophical realism and its relationship to scientific theory, this chapter eschews broader claims about the `truth' of science in favor of more narrowly arguing for the rejection of a troubling, albeit common, position, namely, that economic theory and modeling should not and can not ``aspire toward purposefulness and... practical use'' \autocite[35]{rubinstein2012}. This \textit{theory-as-fable} view is deeply worrying. If correct, the standard by which we---scientists, philosophers, the general public, etc---are to evaluate economic models is altogether divorced from their ability to materially improve our lives. Thankfully, I believe this position is not supported by the available evidence. To borrow jargon from a familiar philosophical argument for realism: it would indeed be miraculous if the successes of design economics rested on a body of theory that amounted to a mere collection of fables. This is the central thesis of the present chapter.

This argument specifically applies to the subdomain of design economics within the broader discipline of economics. In the past few decades, the profession of economics has become increasingly involved in the design of institutions and markets. The ascendancy of design economics is well-documented across the economics profession, as well as outside of it\footnote{See \autocite{roth2019,spiegler2024} for a contemporary view of this rise by economists. From a complementary perspective in philosophy of science, see \autocite{alexandrova2009,guala2001}.}. Part of this rise has undoubtedly been due to some very high-profile success stories; examples that seem to indicate the application of microeconomic theory leads to some kind of empirical success. Awards like the Nobel Prize and the John Bates Clark medal have been given to economic theorists working in associated theoretical disciplines of mechanism design and matching theory on the basis of their technical contributions and ``how basic research can subsequently generate inventions that benefit society'' \autocite{nobel2020}.

Despite the widely heralded success stories of design economics, some economists and philosophers of science are skeptical of the role of economic theory in contributing to these successes. Most notably, renowned microeconomic theorist Ariel Rubinstein has advanced a view of economic theory wherein theory ought not ``engage in predictions or recommendations'' \autocite[36]{rubinstein2012}. On this view, economic theory is merely a collection of fables. This position is radical: fables do not yield predictions nor offer scientific insight or intuition. Even more forcefully, Rubinstein \autocite*[37]{rubinstein2012} asserts he is ``obsessively occupied with denying any interpretation contending that economic models produce conclusions of real value.'' It is my belief this view is entirely wrongheaded.

My argument adapts a familiar philosophical argument for scientific realism to the domain of economics. The \textit{no-miracles} argument for scientific realism \autocite{putnam1975} probabilistically infers the truth of a scientific theory from its success. This argument is particularly well-suited to explain the success of a natural science like physics; in the context of economics, some work is required to adapt realist concerns to the local domain of economics. To advance a minimal non-fable account of economic theory in the domain of design economics I propose a modified (weaker) \textit{explanatory challenge} version of the no-miracles argument with a similar non-miraculous inference. Insofar as successes accrue to a scientific domain it becomes increasingly unlikely that the successes cannot be explained by a common feature that is not reducible to random chance. This version of the no-miracles argument is designed for scientific domains like economics where the truth of a scientific theory is difficult to establish.

Armed with the explanatory challenge version of the no-miracles argument, it is then a matter of determining the kinds of features that explain the success of design economics. Some obvious candidates emerge. The role of experiment and computation have been heralded as ``natural complements'' \autocite[1342]{roth2002} to theory and there is broad consensus between economists and philosophers of science concerning their importance in bringing about success. Additionally, personal and professional relationships between advising economists and policymakers are also considered. These are also widely considered important. What of economic theory and its role in explaining the success of design economics? Here, views diverge drastically. Some assert that the case for theory is wildly overblown \autocite{nikkhah2008,rubinstein2012}. Others take a more moderate view, believing it is useful in conjunction with other features like experiment and computation, as well as being circumscribed by point of application \autocite{alexandrova2009}. Finally, some even assert theory has positive predictive value or sharpens the intuitions of economists entirely on its own \autocite{roth2019}. 

This final story concerning the role of theory in honing economists' intuitions is by far the most common. It is even called the ``mainstream paradigm for market design'' \autocite[10]{sönmez2023minimalist} by those adjacent to it. It supports the view that the role of theory in explaining the successes of design economics is not reducible to random chance. Furthermore, a particular quality of theories of design economics supports this position. The \textit{projective} quality of these theories is such that the world can be made to reflect or ``mirror'' \autocite{guala2001} these theories. Theories in design economics such as, say, auction design, model choices designers subsequently take. The world can then be made to look like the model. Taken together, these claims support the probabilistic inference that theory explains the successes of design economics. The alternative theory-as-fable view makes a miracle of this success.

This chapter is structured as follows. In section \ref{fable_sec_explain} I give a brief overview of philosophical debates in scientific realism, paying particular attention to the no-miracles argument for scientific realism. A modified, explanatory challenge version of the no-miracles argument is introduced in the specific context of the scientific domain of economics. Since this argument begins from the observations of a science's successes, section \ref{fable_sec_design} introduces the science of design economics and highlights its successes, showcasing two notable success stories: the residency matching algorithm developed for new doctors in the United States and the auction design implemented by the FCC for their radio spectrum auctions. A consideration of features that might account for the successes of design economics then takes place in section \ref{fable_sec_success}. Here, experiment, computation, personal and professional relationships, as well as theory are all considered, and arguments for and against their role in explaining success is the explored. There is broad consensus on the role played by all features excluding theory. Theory is considered in detail in section \ref{fable_sec_theory}. I find the theory-as-fable view untenable in light of both the testimonies of those who served as economic advisors on the successful instances of design economics and the projective quality of the theory. Finally, I conclude in section \ref{fable_sec_conc}.





\section{Explaining a Successful Science}\label{fable_sec_explain}

% {\color{red}
% \begin{itemize}
%     \item Why couldn't we be non-fable instrumentalists? Why \textit{adapt} realism instead of giving it up altogether?
% \end{itemize}
% }

What should we make of a successful scientific practice? Are we to believe its theoretical premises? Its ability to be successful again? These questions have been at the heart of philosophy of science's attempt to grapple with the philosophical doctrine of \textit{scientific realism}---the idea that the theoretical content of a successful science is, in some sense, ``true''---since the dawn of the Twentieth Century. Much of this preoccupation has centered on the scientific discipline of physics, which boasts a host of impressive scientific achievements. On the other hand, economics, the ``dismal science'', hasn't yet split the atom or found a way to reliably mitigate the woes of inflation or unemployment, let alone forecast them. However, design economics has slowly accrued a number of success stories that it can rightfully be proud of\footnote{These are covered in detail in Section \ref{fable_sec_design} below.}. But how can we adapt considerations of realism to social scientific domains like economics? Answering this question is the focus of the present section.

It is helpful to sketch the philosophical doctrine of scientific realism in further detail. This story is most straightforwardly understood in the context of physics; I will subsequently address realism in the context of economics. Typically, scientific realism is understood as the philosophical position that something about a science---its theories, structures, entities, experiments, etc---is true. The notion of `truth' alluded to is up for debate. Sometimes a literal, semantic notion of truth is invoked whereas other times it suffices that a scientific theory is only ``approximately true'' (see \cite{chakravartty2017} for discussion). Furthermore, ``[e]ven when our sciences have not yet got things right, the realist holds that we often get close to the truth. We aim at discovering the inner constitution of things and at knowing what inhabits the most distant reaches of the universe'' \autocite[21]{hacking1983}. Realists maintain that (1) there is a mind-independent reality and (2) our science has access to it. Taken together, a generic realist philosophical position towards science amounts to: ``our best scientific theories give true or approximately true descriptions of observable and unobservable aspects of a mind-independent world'' \autocite[\S 1.2]{chakravartty2017}.

In contrast to scientific realists, \textit{scientific anti-realists} believe that ``[scientific theories] are tools for thinking. Theories are adequate or useful or warranted or applicable, but no matter how much we admire the speculative and technological triumphs of natural science, we should not regard even its most telling theories as true'' \autocite[21]{hacking1983}. On this view, theoretical entities like electrons or quarks are convenient fictions that allow scientists to better understand the world. The same goes for theories like the theory of general relativity. It is not to be interpreted as a literally true description of the external world but as a tool for thinking. There are simply better and worse tools; none of them are true in any sense of the word. This division between realism and anti-realism is not a disagreement about whether something (a practice, theory, etc) is or isn’t scientific, or whether some science actually works, but instead about whether the epistemological apparatus of science actually corresponds to something in the external world (i.e., it is “true” or “approximately true”).

There are as many arguments for scientific realism as there are against it (for anti-realism). One of the most prominent defenses of scientific realism follows from the intuition that it would be unlikely if a successful science wasn't, in some sense, true. The more a science is successful, the more unlikely it would be if its underlying theory didn't latch onto something in the external world. This is known as the \textit{no-miracles argument} for scientific realism. On this view, realism ``is the only philosophy that doesn't make the success of science a miracle'' \autocite[73]{putnam1975}. A rough schematization of the argument is as follows:
\begin{align*}
    & P1 \quad \text{Science $X$ is \textit{successful}} \\
    & P2 \quad \text{Science $X$ has theoretical content $Y$} \\
    & P3 \quad \text{If $Y$ weren't \textit{true}, the successes of $X$ would be \textit{miraculous}} \\
    & C1 \quad \text{$Y$ is true}
\end{align*}
\noindent Where the relevant notions of \textit{success}, \textit{truth}, and \textit{miracle} require further explication. Different realist positions can be articulated on the basis of different understandings of these notions, as well as the links between them. Of all the arguments for scientific realism; however, the no-miracles argument puts the successes of a given science front and center: in the absence of such successes, no account of the ``truth'' of the scientific theory can be given.

By way of an example, we can consider a no-miracles argument for the truth of the theory of gravity in virtue of its correct prediction of the orbit of Halley's comet. Halley's comet is a short-period comet that is visible to the naked eye from Earth on average once every 76 years \autocite{nasa_halley}. The theory of gravity developed by Isaac Newton was used by his contemporary Edmond Halley to correctly predict the next time the comet would be visible in his \textit{Synopsis of the Astronomy of Comets} (1705). This correct prediction (success) is used to argue that the theory of gravity is true. A single correct prediction can be augmented by other successful predictions, each serving to render the ``miraculous'' inference increasingly unlikely. Of course, many philosophers of science committed to antirealist philosophical positions have argued against inferring the truth of a scientific theory from its success (see, for example, \cite{stanford2000}); this example merely illustrates how a no-miracles argument proceeds. First, the theoretical component of a science is identified and its successes are established. Then, the truth of the theory is established in virtue of the likelihood that the successes weren't due to random chance.

The no-miracles argument is designed to argue for the truth of a scientific theory in virtue of that theory's purported ability to generate scientific successes. In the context of economics, however, it makes little sense to speak of the truth of an economic theory in such frankly metaphysical terms. For starters, the idea that a fully rational, self-interested agent is a literally true description of anyone is a ``non-starter'' \autocite[328]{alexandrova2009}. Noted economists like Milton Friedman \autocite*{friedman1953} have famously made a virtue of unrealistic assumptions and their role in economic theory. Truth---in any form---is clearly not something can be inferred from any success in this scientific domain. How then should we interpret successes in the field of economics? It is helpful to reconsider the no-miracles argument. This argument can also be viewed as creating an \textit{explanatory challenge} where the successes of a given science merits an explanation. What accounts for the successes? In the context of a science like economics, the goal is to yield a suitable explanation which does not make successful scientific practices unlikely.

The explanatory challenge argument begins, like the no-miracles argument above, from the observation that a science is successful. Instead of narrowly considering the role of theory in achieving that success (and its subsequent truthfulness) this argument allows for arbitrary features of the science to take the place of theory:
\begin{align*}
    & P1 \quad \text{Science $X$ is \textit{successful}} \\
    & P4 \quad \text{Science $X$ has common feature $F$} \\
    & P5 \quad \text{It would be \textit{miraculous} if $F$ didn't explain the success of $X$} \\
    & C2 \quad \text{$F$ explains the success of $X$}
\end{align*}
\noindent Here, it is still necessary to both explicate a version of \textit{success} and expound a probabilistic case for what constitutes a \textit{miracle}. Notice, however, there is no notion of truth lurking in the conclusion. This argument is much weaker than a no-miracles argument for scientific realism. The challenge here is, unlike in the no-miracles argument for philosophical realism, merely to argue that some $F$ contributes to the success of science $X$. There is no abductive-type argument for why future $F$ is more important than any other feature $F'$ or that $F$ cannot contribute absent this other feature. Instead, the goal is simply to argue that it is unlikely feature $F$ does not play a role in explaining\footnote{The argument is robust to all philosophical variants of ``explanation’’ (see, for discussion, \cite{woodward2021}). Informally, ``explanation’’ is used here to mean ``contributes to the success of science’’.} the success of science $X$.

It is helpful to unpack this argument in more detail. What constitutes an adequately circumscribed ``science''? This question is not intended to raise the specter of the problem of scientific demarcation. Instead, it can be read as asking what merits the consideration of design economics independently from the rest of economics. Note, I do not consider design economics a separate science. As I understand it here, design economics is a collection of institutional design problems (concerning auctions, matching markets, etc). It is as much a sociological phenomenon as it is anything else: it is the collection of institutional design problems that have attracted members of the academic economics profession\footnote{Note, insofar as there are other design problems which have not attracted academic economists (say, for example, in the design of public transport networks, a problem that usually attracts civil engineers) they are not considered problems of design economics as investigated here.}. I am not concerned with whether this collection of problems is a distinct science. My goal is to explain why this collection of institutional design problems that have attracted academic economists has exhibited success. The use of ``science'' in the premises above is merely intended to collect the relevant success stories of design economics: these are the explananda that merit an explanans. 

What of ``features''? This framing is deliberately constructed to avoid the narrow focus on theory alone. The animating idea is simple: aspects of a science like its use and reliance on experiment and computation may also explain its success. In the case of design economics, for example, experimental work was conducted to assess bidders' behavior in controlled environments prior to the final version of the 1994 FCC auctions \autocite{plott1997} and experiments were used to confirm field observations concerning the stability of different matching algorithms \autocite{kagel2000}. I will return to this point in more detail below. Crucially, however, I take the relevant features that explain a science's success to be those that are not reducible to random chance. If the collection of successful cases that are to be explained all took place on a Tuesday, this would not make the feature `took place on a Tuesday' the relevant kind of explanation. This is because the explanation is not necessary to explain a science's success. Whereas the original no-miracles argument establishes that the truth of a scientific theory is sufficient to explain its success (how else could we infer a miracle if the theory wasn't true?) the weaker explanatory challenge version merely asserts that a feature is necessary for explaining success. 

Finally, how should we reason about the probabilistically about the nature of a ``miracle''? The original formulation of the no-miracles argument has been criticized by philosophers of science as an instance of fallacious reasoning known as the `base rate fallacy' \autocite{callender2004}. If the set of candidate theories from which a given theory is drawn overwhelmingly contains true theories, then the conditional probability the theory is true given it is successful is obviously very high. Similarly, if the likelihood of a given theory being true is low, then the resulting conditional probability is also low. Thus, ``the no-miracles argument turns on neglecting this base rate'' \autocite[326]{callender2004}. This contention challenges the nature of what constitutes a miracle with respect to the success of a science.

In the present context, there are two issues the base rate objection. First, it hinges on a restricted understanding of scientific truth: a scientific theory is either true or false. Insofar as a scientific theory cannot be characterized so starkly, this contention loses its force. Second, the notion of ``miracle'' required to establish that a common feature explains the successes of science can be significantly weakened without undermining the conclusion. As the number of successful instances of a scientific endeavor grows, insofar as they share common features, these features are increasingly \textit{unlikely} to fail to explain the success. The colloquial use of the word ``miracle'' merely stands in for a probabilistic argument concerning whether a common feature of the science explains its successes. The weakened no-miracles argument requires identifying features that are not reducible to random chance; the issue at hand is whether a plausible story can be given for how these common features explain a science's successes. The objective of this argument is to show that, as the number of successful cases grows, common features they share explain the successful cases.

The explanatory challenge version of the no-miracles argument adapts the philosophical debate over scientific realism to scientific domains where a notion of truth is harder to establish. This project is connected to other attempts by philosophers of science to establish \textit{local realisms} specific to scientific domains like economics (see \cite{maki2009}). The key conviction that animates this section is that there should exist some basis to ascertain whether a science that is not in the business of truth-telling is more or less `real', in the sense of latching onto a mind-independent reality. If a scientific domain like economics can muster success stories that, say, reliably improve welfare or maximize revenue or result in efficient outcomes, it ought to be possible to dignify the common features of these successes with the recognition they deserve. 






% \section{Realism and `No-Miracles' Arguments}

% Scientific realism is the philosophical position that something about a science---its theories, structures, entities, experiments, etc---is true, in some sense of the word. More generally,
% \begin{quote}
%     ``\textit{Scientific realism} says that the entities, states and processes described by correct theories really do exist. Protons, photons, fields of force, and black holes are as real as toe-nails, turbines, eddies in a stream, and volcanoes. The weak interactions of small particle physics are as real as falling in love. Theories about the structure of molecules that carry genetic codes are either true or false, and a genuinely correct theory would be a true one.
    
%     Even when our sciences have not yet got things right, the realist holds that we often get close to the truth. We aim at discovering the inner constitution of things and at knowing what inhabits the most distant reaches of the universe. Nor need we be too modest. We have already found out a good deal.'' \autocite[21]{hacking1983}
% \end{quote}
% \noindent There is no singular or unified account of scientific realism. It is possible to be a realist about scientific theories (e.g., quantum mechanics) or a realist about unobservable entities (e.g., protons or electrons). It is possible to believe that scientific theories are literally, semantically true or that they are only `approximately' true, for some notion of approximation. 
% Most commonly, however, ``[s]cientific realists seek to establish a link between theoretical truth and predictive success, suitably understood'' \autocite[p29]{saatsi2011}.

% Despite these differences, one can broadly identify three realist commitments \autocite{chakravartty2017}. These are (1) the existence of a mind-independent world investigated by the sciences; (2) a literal interpretation of scientific claims (i.e., taking claims at `face value'); and, (3) that theoretical claims constitute knowledge of the world. Taken together, a generic realist philosophical position towards science amounts to: ``our best scientific theories give true or approximately true descriptions of observable and unobservable aspects of a mind-independent world'' \autocite[\S 1.2]{chakravartty2017}. Note, so varied are the stripes of realism that even this broad characterization could still be considered contentious.

% In contrast to this position, 
% \begin{quote}
%     ``\textit{Anti-realism} says the opposite: there are no such things as electrons. Certainly there are phenomena of electricity and of inheritance but we construct theories about tiny states, processes and entities only in order to predict and produce events that interest us. The electrons are fictions. Theories about them are tools for thinking. Theories are adequate or useful or warranted or applicable, but no matter how much we admire the speculative and technological triumphs of natural science, we should not regard even its most telling theories as true.'' \autocite[21]{hacking1983}
% \end{quote}
% \noindent This division between realism and anti-realism can be construed less as an agreement about whether something is or isn't scientific, or whether some science actually works, and instead as a disagreement about whether the epistemological apparatus of science actually corresponds to something in the external world (i.e., it is ``true'' or ``approximately true'').

% There are as many arguments for scientific realism as there are against it (\textit{for} anti-realism). In this chapter, I am concerned with developing a `no-miracles argument' for scientific realism in the context of design economics. No-miracles arguments for scientific realism claim that realism ``is the only philosophy that doesn't make the success of science a miracle'' \autocite[73]{putnam1975}. The powerful intuition that motivates this argument is straightforward: if a science is widely successful, it is unlikely that its theoretical content isn't, in some sense, true. The argument is structured as follows:
% \begin{align*}
%     & P1 \quad \text{Science $X$ is \textit{successful}} \\
%     & P2 \quad \text{Science $X$ has theoretical content $Y$} \\
%     & P3 \quad \text{If $Y$ weren't \textit{true}, the successes of $X$ would be \textit{miraculous}} \\
%     & C1 \quad \text{$Y$ is true}
% \end{align*}
% \noindent Where the relevant notions of \textit{success}, \textit{truth}, and \textit{miracle} require further explication. Different realist positions can be articulated on the bases of different understandings of these notions, as well as the links between them. Of all the arguments for scientific realism; however, the no-miracles argument puts the successes of a given science front and center: in the absence of such successes, no account of the ``truth'' of the scientific theory can be given.

% By way of an example, we can consider a no-miracles argument for the truth of the theory of gravity in virtue of its correct prediction of the orbit of Halley's comet. Halley's comet is a short-period comet that is visible to the naked eye from Earth on average once every 76 years \autocite{nasa_halley}. The theory of gravity developed by Isaac Newton was used by his contemporary Edmond Halley to correctly predict the next time the comet would be visible in his \textit{Synopsis of the Astronomy of Comets} (1705). This correct prediction (success) is used to argue that the theory of gravity is true. This single correct prediction can be augmented by other successful predictions, each serving to render the ``miraculous'' inference increasingly unlikely. Of course, many philosophers of science committed to antirealist philosophical positions have argued against inferring the truth of a scientific theory from its success (see, for example, \cite{stanford2000}); this example merely illustrates how a no-miracles argument proceeds. First, the theoretical component of a science is identified and its successes are established. Then, the truth of the theory is established in virtue of the likelihood that the successes weren't due to random chance.

% The structure of this argument has led some philosophers to contend that the no-miracle argument is an instance of fallacious reasoning known as the `base rate fallacy' \autocite{callender2004}. If the set of candidate theories from which a given theory is drawn overwhelmingly contains true theories, then conditional probability the theory is true \textit{given it is successful} is obviously very high. Similarly, if the likelihood of a given theory being true is low, then the resulting conditional probability is also low. Thus, ``the no-miracles argument turns on neglecting this base rate'' \autocite[326]{callender2004}. This contention challenges the nature of what constitutes a miracle with respect to the success of a science. However, it hinges on a restricted understanding of scientific truth: a scientific theory is either true or false. Insofar as a scientific theory cannot be characterized so starkly, this contention loses its force, {\color{red}a point that will be made more forcefully in the context of economic theory below}.

% While the no-miracles argument can be given as an argument for the philosophical position of scientific realism, it can also be more weakly understood as posing an {\color{red}`explanatory challenge'}\footnote{I would like thank Kevin Zollman for this helpful framing.} where an explanans is sought for the explanandum of a successful science. The challenge is simply: what might explain a given science's success. In the context of design economics explored below, this is equivalent to granting premises $P1$ and $P2$ above (but not $P3$) and then asking `What role does theoretical content $Y$ play in the success?' Note, this construal is especially helpful in the context of microeconomics, where dominant accounts of the importance of microeconomic theory emphasize its role as a `fable' and the related roles of economic theorists as `tellers of fables' \autocite{rubinstein2006,rubinstein2012}. On this understanding of the no-miracles argument, it invites explanations for the success of a given science that are not reducible to random chance. On this view, the truth of a scientific theory is but one possible explanans for the success of a given science and must be explicitly argued for.

% The explanatory challenge version of the no-miracles argument retains premises $P1$ and $P2$ (in modified form) above but adds a new premise $P4$ and conclusion $C1$. In its most general guise, stripped of a narrow focus on scientific theory, it proceeds as follows:
% \begin{align*}
%     & P1 \quad \text{Science $X$ is \textit{successful}} \\
%     & P2 \quad \text{Science $X$ has common feature $F$} \\
%     & P4 \quad \text{It would be \textit{miraculous} if $F$ didn't explain the success of $X$} \\
%     & C2 \quad \text{$F$ explains the success of $X$}
% \end{align*}
% \noindent Here, it is still necessary to both explicate a version of \textit{success} and expound a probabilistic case for what constitutes a \textit{miracle}. Moreover, this restructured argument is designed to function if a suitable feature $F$ can be argued for. The challenge here is, unlike in the no-miracles argument for philosophical realism, to argue that some $F$ contributes to the success of science $X$. There is no abductive-type argument for why future $F$ is \textit{more important} than any other feature $F'$ or that $F$ cannot contribute absent this other feature. Instead, the goal is simply to argue that is is unlikely feature $F$ does not play a role in explaining the success of science $X$.

% The intuition behind this formulation is straightforward. As scientific practices that attain clear and well-defined notions of success accumulate, it becomes increasingly unlikely (``miraculous'') that their common features do not explain their success. Admittedly, there may be shared features that are nonetheless reducible to chance. If all the successes were conducted on a Tuesday, this would be unsatisfactory with respect to explaining why the cases were ultimately successful. However, as the number of cases grows the likelihood of a feature reducible to random chance being shared across all cases diminishes. The remaining challenge is to articulate why the shared features are plausible candidates for explaining the success of the collection of cases. 

% There are a number of notable features of this repurposed no-miracles argument. First, the common features $F$ are by no means restricted to the common philosophical preoccupation of scientific theory. Experimentation and computation may also be shared across successful cases and can be argued to explain their success. A corollary of this is that there is no reference to the ``truth'' of a feature $F$ in this formulation. The goal is \textit{explaining} the success of a collection of scientific practices, not arguing for the truth of their theoretical content\footnote{\color{red}TODO how deep down the `explaining' rabbit hole do I need to go?}. Finally, note this formulation is not susceptible to the base rate fallacy, since we are no longer restrictively considering the truth-value of a feature (what does it mean for an experiment or computation to be true?) 

% Ultimately, The explanatory gap version of the no-miracles argument is much weaker\footnote{\color{red}TODO think about: (1) if $F$ is in all the failures; (2) conjunction of $F_1 + F_2$ is doing the work; (3) successful cases without $F$} than the original argument for scientific realism. This is by design. The goal of this form of argument for scientific realism is placing candidate feature $F$ in contention for explaining the success of instances of a given scientific endeavor. This form of argument is constructed to insulate against the conclusion that, should feature $F$ be common to successful cases of science $X$, it doesn't explain their success. This conclusion---the rejection of C2 (above)---is structurally analogous to arguments for the consideration of economic theory as a collection of fables.







\section{What is `Design Economics'?}\label{fable_sec_design}

`Design economics' is typically understood as ``the part of economics intended to further the design and maintenance of markets and other economic institutions'' \autocite[p1341]{roth2002}. It is often used synonymously with the designation `market design' (as in \cite{roth2018}) although it is not limited in applicability to markets and marketplaces; the label also covers auction design, hiring practices, organ exchanges, and matching students to schools. At its core, design economics models the choice a principal (designer) makes concerning how to optimally, efficiently, or equitably design an institution. It is not limited to a single theoretical approach instead drawing (not exclusively) from social choice theory, mechanism design, and matching. Design economics aims to inform policy and other concrete decisions concerning questions of institutional design; it has a practical focus that distinguishes it from other branches of economic theory. 

This practical focus entails that experimental and computational economics are ``natural complements'' \autocite[p1342]{roth2002} to theory and thus, market design calls for an `engineering approach', which noted economist Alvin Roth \autocite*[p1342]{roth2002} has elaborated by way of an analogy with the construction of suspension bridges:
\begin{quote}
    The simple theoretical model in which the only force is gravity, and beams are perfectly rigid, is elegant and general. But bridge design also concerns metallurgy and soil mechanics, and the sideways forces of water and wind. Many questions concerning these complications can’t be answered analytically but must be explored using physical or computational models. These complications, and how they interact with the parts of the physics captured by the simple model, are the domain of the engineering literature.
\end{quote}
\noindent The key idea behind the engineering approach is that computation and experimentation ``fill[] the gaps between theory and design'' \autocite[p1374]{roth2002}. Computational methods analyze settings that are too complex to solve analytically and laboratory experiments offer predictions about how people will behave in these environments.

% what differentiates design economics from other approaches? (1) non-ironic approach, (2) designed vs naturally occurring phenomena \autocite{spiegler2024}

In this chapter, I will cover two important case studies that are commonly considered \textit{the} success stories of design economics: (1) the design of a labor clearinghouse for American doctors and (2) the design of the US Federal Communication Commission's (FCC) auction for different parts of the radio spectrum. Although both of these cases occurred in the 1990s, the formative decade where economists were presented with opportunities to help design the rules for complex markets for the first time, the intellectual origins of design economics can be traced to the 1960s and 70s. Early work on auctions \autocite{vickrey1961} and matching mechanisms \autocite{gale1962} blossomed into substantial theoretical literature in mechanism design and matching theory. Mechanism design, in the spirit of Hurwicz \autocite*{hurwicz1972}, became a discipline in its own right, which now boasts its own specialized journal, the \textit{Review of Economic Design} (created in 1994). A similar story exists for the development of matching theory. Ultimately, both of these disciplines blossomed alongside related empirically oriented literature on the outcomes of particular forms of institutional design\footnote{See, for example, discussions by Roth \autocite*{roth2002} and Myerson \autocite*{myerson2008}.}

It is worth briefly highlighting the achievements of the 1990s since they represent the first significant achievements of design economics. Interested readers can consult \autocite{roth2002} for a longer discussion. First, we begin with the example of the market for new American doctors.

\begin{example}[Entry-Level Labor Market for American Doctors]
    The entry-level position for an American doctor is called a \textit{residency}. A good residency substantially influences the career paths of doctors and residents provide much of the labor force of hospitals. In the 1940s, competition for people and positions was so fierce it led to market failure, where the market `unraveled' and students were being appointed to jobs a full two years before graduating from medical school \autocite[p1346]{roth2002}. Thus, students were hired before much indication of their medical school performance was known to their employer, a major source of inefficiency. This market failure was resolved in the 1950s by the introduction of a centralized clearinghouse called the National Resident Matching Program (NRMP) where students would submit rank-ordered lists of jobs and an algorithm was devised to produce a matching of students with hospitals. By 1951 (and lasting into the 1970s) over 95\% of positions were filled through this match.

    However, by the 1990s, there was a crisis of confidence in the labor market. The issue was that the original algorithm devised by the NRMP was designed to match individuals and not married couples, who were increasingly graduating together from medical school. The key problem for the NRMP was that the matches generated by the original algorithm were not `stable': there were pairs of individuals and hospitals who were not matched to each other that would prefer to be matched instead of their proposed matching. Empirical evidence suggests that stability is an important criterion for a successful clearinghouse. Thus, two economists \autocite{roth1999} proposed an alternate design of the clearinghouses' algorithm which would generate a stable match for both individuals and couples. The design was completed in 1996 and implemented in 1997. It is used in a number of medical labor market clearinghouses across the world to this day.
\end{example}

\noindent The other major success story of the 1990s was the design of the FCC's spectrum auctions.

\begin{example}[FCC Spectrum Auctions]
    In 1993 the United States (US) Congress directed the Federal Communications Commission (FCC) to design auctions to efficiently allocate radio spectrum licenses and raise money for the government. Until 1981, spectrum licenses had been historically allocated through a political process called ``comparative hearings'', whereas after 1981 they were allocated by lottery. Both procedures were characterized by lots of rent-seeking behavior and bureaucratic complications, leading to substantial delays, all of which Congress wanted to avoid. A further issue was that auctions for spectrum licenses had already been tried in other countries and, in the notable case of Australia, evidenced `gaming' where participants submitted multiple highest bids and withdrew their winning bids to acquire the license at a lower cost after the auction concluded. Since the FCC was concerned with mitigating this kind of behavior, they hired academic economist John McMillan (then at UCSD) to advise their staff. Additionally, academic economists were hired by communications companies and put forward proposals on behalf of their clients, many of which were adopted by the FCC.

    The resulting auction was carefully designed to avoid a number of pitfalls that concerned the FCC. The auction was open-bid and multi-round, allowing bidders to get a sense of what other bidders thought the licenses were worth (``price discovery''). This was designed to avoid the problem of the ``winner's curse'' where the bidder who overestimates the value of a license is more likely to submit the winning bid. To overcome the problem of complementarities---where acquiring licenses in combination can change their overall value---the idea of auctioning licenses one at a time was rejected. Additionally, to prevent bidders from concealing their intentions by delaying their bids, the FCC imposed an `activity rule' that required bidders to continually bid on licenses they were interested in or lose their ability to do so in the future. These auctions raised an estimated \$230 billion dollars in revenue for the US government by 2023 \autocite{brookings2023} and have been adopted all over the world.
\end{example}

\noindent These examples highlight the kinds of success that design economics can lay claim to. In the case of residency matching, matches that are stable for a large number of market participants is an example of what constitutes success. In the case of the spectrum license auctions, success concerned efficient auctions that result in substantial government revenue. These are not the only goals that can be designed for. Equity or welfare maximization can also be the targets of design. Moreover, the adoption of the approach of design economics has led to other successes across the world. It is important to sketch the extent of these successes since they provide a basis for arguing about which features contributed to the success.

As a point of departure, it is worth noting the NRMP and FCC continue to, respectively, use matching algorithms and auctions to this day. The original work of Roth and Peranson \autocite*{roth1999} that developed a matching algorithm for matching new doctors has placed over 20,000 doctors per year and has been extended to many other medical fields as well as the market new lawyers (see \cite[1346]{roth2002}). The FCC continues to use auctions to allocate its radio spectrum, most notably again in 2016-7 where the FCC ran an ``incentive auction'', designed to re-purpose the spectrum for new uses, raising \$19.8 billion in revenue for the US government \autocite{fcc_incentive}. Both matching algorithms and auctions have been developed and used across the world. The sale of the British 3G telecom licenses in 2000 was hailed as the ``biggest ever'' auction held on earth, raising \textsterling22.5 billion (approximately 2.5\% of GNP) \autocite{binmore2002}. Dozens of countries across the world have used auctions in increasingly more complicated settings\footnote{See, for example, a recent overview of combinatorial auction designs used in practice \autocite{palacios2022}.}. Matching theory has also been used to rectify the lack of `thickness'---enough buyers and sellers to produce satisfactory outcomes---in kidney exchanges in the United States \autocite{roth2007hbs}. There is too much variation across the institutions that have been designed (auctions, exchanges, matchings, etc), geographic settings, and types of success to fully cover here. What is clear is this: successful practical applications of design economics recur all over the world, where success is broadly understood to cover everything from revenue maximization to stability. 

Before turning to how economics and philosophers of science might address the explanatory challenge posed by the successes of design economics, it is helpful to mention the role of experiment and computation in design economics. As noted above, these features are ``natural complements'' \autocite[1342]{roth2002} to theory and are widely held to account for much of the success of design economics\footnote{For an economist's view of this, see Roth \autocite*{roth2002}. From the perspective of philosophy of science, see Alexandrova and Northcott \autocite*{alexandrova2009}.}. Experimental evidence on the stability of competing market mechanisms (drawn from the experiences of labor markets for doctors in the United Kingdom) was used to argue in favor of the particular matching algorithm adopted by the NRMP \autocite{kagel2000}. In the case of the 1994 FCC auctions, experiments were conducted to investigate how bidder's behavior changes in light of different auction configurations \autocite{plott1997}. In this case, an important aspect of the empirical evidence derived from experiments was to ``teach researchers some of the ways in which it would be safe to perturb the final auction design'' \autocite[320]{alexandrova2009}. Experiments are invaluable tools that economists use to narrow down the range of theoretical concerns that are relevant to a given problem and isolate causal effects.     

Computational aspects of design economics also contribute to its successes in a number of ways. Al Roth \autocite*{roth2002} identifies a number of ways computational techniques contributed to the design of the matching algorithm used by the NRMP (above). Computation was used to explore alternative algorithm designs and their performance on past data. Computation was also used to further generate and support conjectures about the stability of matchings under different configurations of inputs. Thus, computation can be used to ``suggest[] an agenda for theoretical work'' \autocite[1363]{roth2002} just as it can be used to give evidence for or against theories\footnote{This is what motivates chapter \ref{ch_auctions} of this PhD.}. Furthermore, computational methods help us ``analyze games that may be too complex to solve analytically'' \autocite[1374]{roth2002}, offering insights into settings which lack formulation as analytically tractable economic models. Finally, computational methods have also been used to analyze the significance of particular aspects of the FCC's 2016-7 incentive auction \textit{post hoc} \autocite{newman2024}, exploring the performance of alternative auction configurations on realistic models of bidder behavior.

In conclusion, design economics is a part of economics that concerns questions of institutional design, and tackles these questions in a fashion that makes use of experiment and computation to a large degree. It is been likened to a type of engineering, where 
a goal or objective is the target that an economist (engineer) tries to `hit'. This section has outlined two prominent case studies that constitute the exemplary success stories of design economics, as well as sketch the extent to which they have been successfully replicated and extended around the world. There are large differences in opinion concerning how economists and philosophers interpret these successes. The next section covers how commentators have chosen to make sense of these successes.





\section{Explaining the Success of Design Economics}\label{fable_sec_success}

What explains the successes of design economics? The preceding section was designed not only to showcase two important examples which are the touchstones for discussions concerning design economics' success but also to highlight how these successes have been replicated and extended across the world. While individual instances of success have been contested\footnote{I will discuss below the work of Edward Nik-Shah \autocite*{nikkhah2008} in uncovering the suppressed role of commercial interests in the FCC's 1994 spectrum auction from archival records.} the presentation above paints a more general picture of the successes of design economics: the varied goals that have been achieved, the extent of the successes, and the geographic scope of their implementation. It is this higher-level view of the successes of design economics that merits an explanation. This section covers how philosophers of science and economists would react to the explanatory challenge version of the no-miracles argument sketched above. 

Before turning to possible explanations for the successes of design economics a negative conclusion must first be dismissed, namely, that the examples solicited in the preceding section are not examples of success. The animating idea of this objection is something along the lines of `things could have gone better': auctions could have been more efficient, matches more stable, etc (for an example of this kind of criticism, see \cite{ledyard1997}). There are two rejoinders to this objection. First, examples of abject failures are well-documented. Examples like the 2000 Swiss UMTS auction \autocite{wolfstetter2001} or the 1990 New Zealand spectrum auction \autocite[\S1.2.2]{milgrom2004} are clear examples of failure. In the latter case, the outcome was inefficient, raised far less revenue than projected, and the final allocation outcome was deemed ``unnecessarily random'' \autocite[12]{milgrom2004}. In the former case, widely described as ``flop'', far fewer bidders participated than was expected, resulting in a much lower revenue for the Swiss government. The examples presented in the previous section are very clearly not instances of this kind of failure. To what extent they are successes can be explored after the fact with simulations (see \cite{newman2024}). The key insight is that, in the case of failures, alternative auction designs could have performed much better. Moreover, this fact is widely acknowledged. Insofar as it is unclear which designs could have performed better and we avoid obviously negative outcomes like few bidders, unstable matches, and low revenue (etc) these cases should be considered successes. 

Additionally, it is possible to contend that although the previously canvassed examples are indeed successful, no explanation for their success is warranted. This is akin to asserting that their success is, effectively, random. A direct implication of this `no explanation needed' view is that there is no feature (not reducible to random chance\footnote{I will refer to features which are not reducible to random chance as \textit{non-arbitrary} features.}) that explains why these cases were successful. The aforegoing presentation of the explanatory challenge version of the no-miracles argument is deliberately constructed to avoid this conclusion. As the set of successful instances of a science grows, the probability that some common, non-arbitrary feature does not explain their success shrinks. In contrast to the problem of base rates \autocite{callender2004}, this formulation of an argument for realism is not narrowly focused on the role of theory and its relation to truth in explaining success. Thus, as the number of successful cases grows, any common, non-arbitrary feature present in these cases is an increasingly promising candidate for explaining their success.

What features might then explain the successes of design economics? Though it is possible to garner a wide variety of candidates, I will focus on three\footnote{There are undoubtedly more, many of paramount importance. However; these are commonly considered in the literature and (1) and (2) serve as preludes to the central claims of this chapter concerning (3) theory.} sets of features: (1) personal and professional connections; (2) experiment and computation; and, (3) theory. The goal is to collect the ideas and opinions of philosophers of science and economists to better understand how these features might explain the success stories of design economics. Again, it bears emphasizing that establishing an explanatory connection between a feature and successful science is not \textit{abductive}: there is no claim that a given feature is \textit{the best possible explanation} for the success. Instead, the goal is merely to make the case that common, non-arbitrary features mattered in bringing about success. Furthermore, there is nothing that precludes the possibility these features only matter in conjunction. That theory might also require experiment, for example, is well within the conclusion of the argument established here. 

The idea that personal connections between economists, industry stakeholders, and policymakers matter in explaining the success of design economics is widely echoed by economists who've served in advisory capacities. Notably, senior economic advisor to the FCC Evan Krewel extensively documents the key actors who were decisive in bringing economists into the FCC's auction design procedure (and their relationships) in the forward to Paul Milgrom's \autocite*{milgrom2004} \textit{Putting Auction Theory to Work}. Even Milgrom's character---his ``integrity'' \autocite[xix]{milgrom2004}---is cited as important in his efforts to persuade the FCC to adopt his design. These narratives concerning the personal relations between academic economists and policymakers are also echoed by other economists who've had similar roles as economic advisors (see, for example, \cite{roth2002, sönmez2023minimalist}). There is no straightforward means of ascertaining the extent of these relationships across all the successful instances of design economics; however, if it turns out that these cases have this feature of personal connections in common, then this should constitute a possible explanation for their success.

That good relations between consulting economists and policymakers is a feature that explains the successes of design economics will surprise no one who has worked in a government or administrative setting. However, the role of corporate interests in shaping the Overton Window of policy options has not gone unnoticed by commentators. Philosopher of science Edward Nik-Shah's unexpurgated account \autocite*{nikkhah2008} of the 1994 FFC spectrum auction from archival records shows the degree to which economists were used to legitimize outcomes that aligned with the incentives of those corporations that hired them. His conclusion could be no less ambiguous: ``\textit{corporate imperatives demonstratively played the decisive role in determining the auction}'' \autocite[89, emphasis original]{nikkhah2008}. Ultimately, the extent to which this constitutes corruption is not relevant to my argument. However, the extent to which this occurred in other instances of success renders it a feature worthy of consideration for explaining that success.

The role of experiment and computation in design economics was already documented above. Economist Al Roth \autocite*{roth2002, roth2018} has made the case that experiment and computation are complements to theory and drawn comparisons between economics and the discipline of engineering in the natural sciences. Philosophers of science have echoed this perspective, most notably Alexandrova and Northcott \autocite*[320]{alexandrova2009}, who persuasively argue one of the roles of experiments was to ``teach researchers some of the ways in which it would safe to perturb the final auction design.'' Like Al Roth, they echo the idea of economists as `engineers', this time drawing analogies between auction design and the ``development of \textit{racing cars}'' \autocite[331, emphasis original]{alexandrova2009}. ``Theoretical knowledge alone is not enough'' they continue, ``teams also have a huge testing programs, analogous to the experimental test beds of the [FCC's 1994] spectrum auction'' \autocite[331]{alexandrova2009}. 

It is hard to assess the ubiquity of computation and experiment across the successful cases garnered above. Clearly, leading economists working in design economics march to a similar tune: experiment and computation are everywhere heralded as important for the design of institutions \autocite{roth2002,binmore2002,sönmez2023minimalist,milgrom2004}. Philosophers of science critical of the prominence of economic theory in attributions of the success of design economics are unified in their endorsement of the importance of experiment in explaining the success of the FCC's 1994 spectrum auction \autocite{alexandrova2009,nikkhah2008}. Again, insofar as experiment and computation are common across the successful cases of design economics, the broad consensus concerning their importance supports the claim they explain that success. And given the claims by economists above, it seems likely this is in fact common to the success stories of design economics.

The role occupied by the features covered so far in explaining the successful instances of design economics would likely be granted by those critical of the narrative concerning the importance of theory in explaining those very same successes. The evidence mustered above establishes that the role of (1) personal and professional relationships and (2) computation and experiment is not \textit{arbitrary}; it is not reducible to random chance. These features are, in some minimal capacity, necessary for success. The picture for (3) theory is much murkier. In the wake of the FCC's 1994 spectrum auction\footnote{This sentiment was also echoed by Binmore and Klemperer \autocite*{binmore2002} after the UK's 2000 3G spectrum auction.} much fanfare was made by academic economists concerning the crucial role played by theory in determining the success of the auction (see, for example, \cite{mcafee1996,mcmillan1994}). This picture has subsequently been contested by a number of philosophers of science \autocite{nikkhah2008,alexandrova2009}. It is worth outlining the relationship between scientific realism and economic theory---in the vein of a `local realism' for economics \autocite{maki2009} advocated above---to better understand how economists and philosophers of science understand the role of theory in the science of economics.

It is helpful to begin with Friedman's \autocite*{friedman1953} seminal essay `The Methodology of Positive Economics' because the conclusion drawn here is sympathetic to his motivations. Furthermore, this represents a canonical understanding of the role of theory in contemporary economics. For Friedman, ``[t]he ultimate goal of a positive science is the development of a ``theory'' or, ``hypothesis'' that yields valid and meaningful (i.e., not truistic) predictions about phenomena not yet observed.'' \autocite[7]{friedman1953}. The notion of novel prediction alluded to here captures the animating idea of realism in the example of Halley's comet used in the preceding section\footnote{Friedman \autocite*[9]{friedman1953} even notes the role that retrodiction plays in establishing a successful economic theory}. For Friedman, the operative notion of success that characterizes the science of economics is \textit{predictive success}. Notably, the success of a theory can be evaluated independently of how unrealistic its assumptions are:
\begin{quote}
    a theory cannot be tested by comparing its ``assumptions'' directly with ``reality''. Indeed, there is no meaningful way this can be done. Complete ``realism'' is clearly unattainable, and the question whether a theory is realistic ``enough'' can be settled only by seeing whether it yields predictions that are good enough for the purpose at hand or that are better than predictions from alternative theories.'' \autocite[41]{friedman1953}
\end{quote}
\noindent Though there are many ambiguities throughout Friedman's writing regarding his use of `realism' and `realisticness' his essay ``is transparently the manifesto of an engineer rather than a scientist'' \autocite[740]{ross2008}. 

This starting point for evaluating the role of theory in explaining the successes of design economics is helpful in the context of design economics. Though Friedman's primary concern is predictive success (what is often called successful \textit{novel prediction}), the outlook is manifestly that of a scientist concerned with using theory to change (improve) the world. His endorsement of economic theory despite its unrealistic assumptions is a clear indication that his understanding of scientific realism is one which is not centered on a notion of truth\footnote{More specifically, Maki notes that Friedman can be read as ``thinking that the assumptions of his theory have a definite truth value---namely, that of false---and that being unrealistic in this sense is a good thing'' \autocite*[179]{maki1992}. This point is incidental to the central claim above: the philosophical realism Friedman can be read as endorsing is one similar to that of the `economist as engineer', except that his focus is on novel prediction and not outcomes like matching stability or revenue maximization.}. Friedman goes further, stating that attempts to make theoretical assumptions more realistic ``is certain to render a theory useless'' \autocite[30]{friedman1953}. Commentators have noted that ``[w]hile ``realism'' is a name for members in a set of philosophical doctrines, ``\textit{realisticness}'' characterizes features of representations''. Thus, Friedman’s essay constitutes a plea for using false theory for instrumental purposes. This means that
\begin{quote}
    the locus of appropriate criticism of any chunk of economics does not mostly lie at the level of general philosophical description of method, but rather at the level of how the method is used and how its use is constrained and what results it produces. \autocite[33]{maki2009}
\end{quote}
\noindent It is then not hard to recover a Friedman-like endorsement of the role of theory in design economics. Though the assumptions of the theory are unrealistic---as noted above, the ascription of literal truth to assumptions of rationality and self-interest is a ``non-starter'' \autocite[328]{alexandrova2009}---what matters about economic theory from the perspective of a doctrine of philosophical realism is what ``results it produces'' (above).

Milton Friedman's views of the role of theory in economics are diametrically opposed to the influential contemporary microeconomist Ariel Rubinstein, who provocatively argues that economic theory is a merely `collection of fables' \autocite{rubinstein2006,rubinstein2012}. This economic \textit{theory-as-fables} view is the foil for the present chapter. On this view, economic theory ought ``not aspire toward purposefulness... [and] does not engage in predictions and recommendations'' \autocite[35-6]{rubinstein2012}. He explicitly contrasts his approach to economic theory\footnote{Rubinstein is principally concerned with economic \textit{models}; however, we can extend his view to economic theory, viewed simply as a collection of models.} with (1) a view of economic theory which aims to make predictions about the real world and (2) a view of economic theory which aims to sharpen an economist’s perception or intuition. By his own admission, Rubinstein is ``obsessively occupied with denying any interpretation contending that economic models produce conclusions of real value'' \autocite[37]{rubinstein2012}. For Rubinstein, the modest goal of a ``teller[] of fables'' \autocite[882]{rubinstein2006} is all an economic theorist should hope for.

Rubinstein's theory-as-fables approach is complemented by economists who maintain ``a heightened awareness of the gap between reality and its representation, coupled with a detached, bemused attitude to this gap'' \autocite[175]{spiegler2024}. This \textit{ironic} reading of economic theory is a product of the distance between the representation and reality: it shies away from ``taking the model seriously'' \autocite[176]{spiegler2024} in the sense of offering policy prescriptions or scientific predictions. Although the economics profession ``has [historically] been willing to sustain the irony-suffused culture of economic theory'' \autocite[176]{spiegler2024}, a more recent anti-irony turn is connected to the rise of design economics:
\begin{quote}
    The increasing appeal of the ``market design'' field lies in its practitioners’ ability to go through the regular motions of an economic-theory exercise while insisting on a straightforward, nonironic connection to an economic reality. The ``economist as engineer,'' as Al Roth \autocite*{roth2002} called it; irony is not meant to be an engineer’s thing. Market design methodology focuses on tightly regulated economic environments whose actors are expected to follow rigid rules. As a result, the gap between model and reality appears small enough to curb the irony impulse. \autocite[177-8]{spiegler2024}
\end{quote}
\noindent In documenting this turn, Spiegler \autocite*{spiegler2024} is correct to point out the aspects of economic behavior these models fail to capture; however, his own account of the ``curb[ing]'' of the irony impulse is, I believe, exactly in line with where I take the successes of design economics to lie, as I will return to below.

How would someone persuaded by the theory-as-fable view respond to the explanatory challenge version of the no-miracles argument? First, one could object to the probabilistic conclusion. There is simply insufficient evidence to establish, in any reasonable form, the ``miraculous'' inference. Unlikely, perhaps; miraculous, no. As the evidence grows (i.e., the number of successful instances of design economics increases) this view would be subject to change. (After all, at some point every unlikely occurrence yields a miraculous interpretation). Note; however, this path is not open to someone who is ``obsessively occupied with denying... that economic models produce conclusions of real value'' (Rubinstein, above) or shies away from ``taking the model seriously'' (Spiegler, above). Alternatively, on the theory-as-fables view, other features---experiment, computation, personal connections, etc---might explain the successes of design economics but theory falls short of the mark\footnote{Note, an endorser of the theory-as-fable view would even have a hard time admitting that theory was useful only when present {in conjunction} with other features.}. This is equivalent to claiming that the effect of theory is reducible to random chance: it has the same status as the feature corresponding to `it took place on a Tuesday'. 

Here is what Rubinstein has to say about the 1994 FCC spectrum auctions:
\begin{quote}
    I personally know some of the people who planned this tender and similar tenders. They are undoubtedly bright and intelligent. They are also people with two feet firmly on the ground. However, to the best of my understanding, they based their recommendations on basic intuitions and human simulations, and not on sophisticated models of game theory. I do not find any basis for claiming that it was game theory that helped them in planning the tender. At most, these advisors were intimately familiar with a specific type of strategic considerations that we often study in game theory. \autocite[125]{rubinstein2012}
\end{quote}
Here, we find evidence for the view that ``basic intuitions and human simulations'' (above) formed the basis for recommendations given by economic theorists to the FCC. Here, not only is Rubinstein rejecting the view that (1) economic theory can serve ``as a basis for making predictions about the real world'' \autocite[34]{rubinstein2012} but also the weaker view (2) that the objective of economic theory is to ``sharpen perception'' \autocite[34]{rubinstein2012}. 

Clearly, on the subject of the role of theory in explaining the successes of design economics, opinions differ widely. Supposing that experiment and computation are common features across the range of successes there seems to be broad consensus that they are non-arbitrary: they genuinely contribute to explaining the success stories of design economics in a manner that is not reducible to random chance. Without denying the importance of these other features in explaining the successes of design economics a case needs to be made for the non-arbitrariness of theory. This is the role of the proceeding section.







% \section{Realism in Economics}

% It is widely accepted that the philosophical doctrine of scientific realism as formulated above is ill-suited to the scientific domain of economics. In the context of economic theory, the notions of literal or semantic truth have little to offer a science that infamously glorifies unrealistic assumptions\footnote{To quote Alexandrova and Northcott \autocite*[328]{alexandrova2009}: ``[n]obody thinks that \textit{Homo economicus} is a literally true description of anyone. So the idea that economic theories could ever be literally true is a non-starter.}. Milton Friedman's essay `The Methodology of Positive Economics' \autocite*{friedman1953} makes a virtue of simple, unrealistic assumptions, claiming that attempts to make theoretical assumptions more realistic ``is certain to a render a theory utterly useless'' \autocite[30]{friedman1953}. Ultimately, claims concerning the \textit{literal} truth of preferences, utilities, games, etc are never to be taken at face value. How then ought we characterize a doctrine of philosophical realism in the context of economic theory? This section argues that much in the original spirit Friedman penned his seminal essay, what matters is how these abstractions and idealizations are used. What makes a theory more or less `real' is whether we can use it to achieve a given end. 

% It is helpful to begin with Friedman's original essay because the conclusion drawn here is sympathetic to his motivations. For Friedman,
% \begin{quote}
%     [t]he ultimate goal of a positive science is the development of a ``theory'' or, ``hypothesis'' that yields valid and meaningful (i.e., not truistic) predictions about phenomena not yet observed.'' \autocite[7]{friedman1953}
% \end{quote}
% \noindent The notion of novel prediction alluded to above captures the animating idea of realism in the preceding section\footnote{Friedman \autocite*[9]{friedman1953} even notes the role that retrodiction plays in establishing a successful economic theory}. For Friedman, the operative notion of success that characterizes the science of economics is \textit{predictive success}. Notably, the success of a theory can be evaluated independently of how unrealistic its assumptions are:
% \begin{quote}
%     a theory cannot be tested by comparing its ``assumptions'' directly with ``reality''. Indeed, there is no meaningful way this can be done. Complete ``realism'' is clearly unattainable, and the question whether a theory is realistic ``enough'' can be settled only by seeing whether it yields predictions that are good enough for the purpose at hand or that are better than predictions from alternative theories.'' \autocite[41]{friedman1953}
% \end{quote}
% \noindent Though there are many ambiguities throughout Friedman's writing regarding his use of `realism' and `realisticness' his essay ``is transparently the manifesto of an engineer rather than a scientist'' \autocite[740]{ross2008}. This outlook will be recovered by the end of this section; however, we must first address a number of criticisms with this view and others that argue for the `realism' of economic theory.

% Is there a tension between accepting unrealistic assumptions and being a philosophical realist about economic theory? The answer to this question hinges on resolving the ambiguities in Friedman's \autocite*{friedman1953} essay. Maki \autocite*[175, emphasis original]{maki1992} notes that ``[w]hile ``realism'' is a name for members in a set of philosophical doctrines, ``\textit{realisticness}'' characterizes features of representations''. Thus, there are multiple possible views concerning the relationship between a philosophical doctrine of realism and a scientific theory's representations. Notably, in the context of the original philosophical doctrine of realism introduced in the preceding section, Friedman can be read as: ``thinking that the assumptions of his theory have a definite truth value---namely, that of false---and that being unrealistic in this sense is a good thing'' \autocite[179]{maki1992}. Thus, Friedman's essay constitutes a plea for using false theory for instrumental purposes. This reading of Friedman is common among many economists\footnote{See Maki \autocite*[192]{maki1992} for discussion.} and is my preferred reading. Ultimately, as noted by Maki, from a realist point of view
% \begin{quote}
%     here is no general problem with unrealistic models with unrealistic assumptions or the method of isolation by idealization. This means that the locus of appropriate criticism of any chunk of economics does not mostly lie at the level of general philosophical description of method, but rather at the level of how the method is used and how its use is constrained and what results it produces. \autocite[33]{maki2009}
% \end{quote}
% \noindent This conclusion again echoes the animating theme of the present section: what matters about economic theory from the perspective of a doctrine of philosophical realism is what ``results it produces'' (above).

% Friedman's focus on results is much at odds with a more recent view among economists who downplay the pragmatic value of economic theory. This economic \textit{theory-as-fables} view is the foil for the present chapter. Most prominently advocated by microeconomist Ariel Rubinstein \autocite*{rubinstein2012,rubinstein2006} who advocates for the view that economic theory\footnote{Rubinstein is principally concerned with economic \textit{models}; however, we can readily extend his view to economic theory, viewed simply as a collection of models.} ought ``not aspire toward purposefulness... [and] does not engage in predictions and recommendations'' \autocite[\color{red}XXX]{rubinstein2012}. He explicitly contrasts his approach to economic theory with (1) a view of economic theory which aims to make predictions about the real world and (2) a view of economic theory which aims to sharpen an economist's perception or intuition. By his own admission, Rubinstein is ``obsessively occupied with denying any interpretation contending that economic models produce conclusions of real value'' \autocite*[\color{red}XXX]{rubinstein2012}. For Rubinstein, the modest goal of a ``teller[] of fables'' \autocite*[882]{rubinstein2006} is all an economic theorist should hope for.

% Rubinstein's theory-as-fables approach is complemented by economists who maintain ``a heightened awareness of the gap between reality and its representation, coupled with a detached, bemused attitude to this gap'' \autocite[175]{spiegler2024}. This \textit{ironic} reading of economic theory is a product of the distance between the representation and reality: it shies away from ``taking the model seriously'' \autocite[176]{spiegler2024} in the sense of offering policy prescriptions or scientific predictions. Although the economics profession ``has [historically] been willing to sustain the irony-suffused culture of economic theory'' \autocite[176]{spiegler2024}, a more recent anti-irony turn is connected to the rise of design economics:
% \begin{quote}
%     The increasing appeal of the ``market design'' field lies in its practitioners’ ability to go through the regular motions of an economic-theory exercise while insisting on a straightforward, nonironic connection to an economic reality. The ``economist as engineer,'' as Al Roth \autocite*{roth2002} called it; irony is not meant to be an engineer’s thing. Market design methodology focuses on tightly regulated economic environments whose actors are expected to follow rigid rules. As a result, the gap between model and reality appears small enough to curb the irony impulse. \autocite[177-8]{spiegler2024}
% \end{quote}
% \noindent In documenting this turn, Spiegler \autocite*{spiegler2024} is correct to point out the aspects of economic behavior these models fail to capture; however, his own account of the ``curb[ing]'' of the irony impulse is, I believe, exactly in line with where I take the successes of design economics to lie, as I will return to below.

% In stark contrast to the irony-infused view of economic theory, some philosophers of science have argued that these mathematical structures are exactly what is real about the science of economics. This view---\textit{ontic structural realism} (OSR)---asserts that what is real about a science is not its entities but its mathematical structures \autocite{ross2008}. This view was originally developed in the context of fundamental physics, and
% \begin{quote}
%      [a]ccording to OSR, individual objects in physics are heuristics, bookkeeping devices that help investigators manipulate partial models of reality so as to stay focused on common regions of measurement from one probe to the next. I contend that the everyday objects in terms of which economic theory is interpreted also have this character. \autocite[741]{ross2008}
% \end{quote}
% \noindent According to the OSR account of economic theory, 
% \begin{quote}
%     what economics has accumulated is understanding of patterns in optimization, maximization, and melioration by systems of elements for which some states of affairs are more valuable than others. The basic objects of economic theory are optimization problems. Of these, the most interesting, both mathematically and in application, are nonparametric ones—that is, games. \autocite[741]{ross2008}
% \end{quote}
% \noindent Here, it is readily sumrised that \textit{game theory} in economics captures the right sort of ontology for economic science: the structures that best can be said to exist (be ``real'') are games. On this view, progress in economics can be understood as coming from a deeper understanding of these abstract structures. While there is undoubtedly ambiguity on this account surrounding the nature of ``structure'', this view abstracts away from particulars of the agents represented in economic models in favor of locating the realism of economic models in the rules that govern their interaction. Realism, on this count, hinges on whether the economic model of a game corresponds to the actual set of rules and constraints that govern the interactions of those who play it. 

% Whereas some philosophers of economics prefer to ``ignore work that is mainly economic engineering'' \autocite[740, cited above]{ross2008}, others choose to orient their arguments for philosophical realism on the basis of our ability to use science for the goals of engineering. An early notable account in this vein is Ian Hacking's argument for \textit{entity realism}---the idea that, although discussions of realism and anti-realism about theories is ``necessarily inconclusive'' \autocite[71]{hacking1983}, 
% \begin{quote}
%     [o]nly at the level of experimental practice is scientific realism unavoidable. But this realist is not about theories and truth. The experimentalist need only be a realist about the entities used as tools. \autocite[71]{hacking1983}
% \end{quote}
% \noindent This view was originally formulated for the science of physics: we are convinced of the reality of electrons because we can manipulate them in service of other engineering goals. The experimentalist believes in the existence of electrons ``because we can use them to \textit{create} new phenomena, such as the phenomenon of parity violation in neutral current interactions'' \autocite[84, emphasis original]{hacking1983}. This view is pithily summarized as `if you can spray them, then they are real' and it is engineering, not theorizing, that yields proof of scientific realism about entities.

% A full discussion of the role of experiments and entity realism is both beyond the scope of this chapter and far from germane to the problem of philosophical realism concerning economic theory in design economics. However, the turn to engineering as the warrant for belief in philosophical realism has been more recently echoed by philosophers of economics studying auction design. Alexandrova and Northcott \autocite*{alexandrova2009} argue that progress in the economics of auction design is akin to progress in engineering. Here, economic theory is ``becoming better at capturing particular aspects of reality'' and although auction models ``are not literally true, nevertheless they do successfully capture real patterns of strategic interaction between bidders'' \autocite[328]{alexandrova2009}. Unfortunately, auction models capture some aspects of reality better than others and there is no general understanding of the seriousness of a given omission outside a specific context. They conclude ``no context-general sense of theoretical progress can be sustained'' \autocite[328]{alexandrova2009}. Thus,
% \begin{quote}
%     the criterion for judging a new piece of theory should be: How much does it help economic engineers achieve empirical success? An expanding and dazzling theoretical repertoire of new methods, heuristics, and categories is, alas, of no empirical value \textit{in itself}. Rather it can only be valuable instrumentally. \autocite[333, emphasis original]{alexandrova2009}
% \end{quote}
% \noindent Armed with this view on the relationship between theory and engineer, their verdict is unequivocal: ``pure research cannot even be said to be pursuing scientific truth'' \autocite[333]{alexandrova2009}. If economic theory has any scientific value, it is \textit{only} in the context of achieving a specific goal. The ``detached, bemused attitude'' \autocite[above]{spiegler2024} of the ironic theorist is no longer one of science.

% The sharp conclusion drawn by Alexandrova and Northcott \autocite*{alexandrova2009} dovetails with the idea of ``local realism'' advocated by Maki \autocite*[9]{maki2009} in the context of philosophy of economics. A realist philosophical doctrine in economics is not hostage to the concerns that animate the truth-laden or entity-focused accounts of physics. Successful applications of theory are clustered around engineering goals: `design economics' is a catch-all term for a collection of institutional design problems where theory has yielded successful outcomes. These outcomes aren't necessarily those of prediction either. As remarked above, the NRMP aimed at stability and the FCC auctions, efficiency. Predictive accuracy was merely a means to an end. Furthermore, design economics has seemed to have shrunk the ``gap'' between representation and reality, inviting a renewed consideration of the realism of the structures at the core of our economic models. With these concluding thoughts in mind, we can now probe in a little more detail exactly what about the theory behind design economics might best account for its success.





\section{The Role of Theory}\label{fable_sec_theory}

The theory-as-fable view of the role of economic theory in explaining the successes of design economics entails a radical conclusion: theory is, effectively, useless. But sustaining this conclusion in the face of the explanatory challenge version of the no-miracles argument is no simple feat. A \textit{fable-ist}\footnote{Equivalently, an adherent of the theory-as-fable view.} cannot contend that theory explains the successful cases only conjunction with other features (for then it would be practically useful and lose its status as a mere fable). Furthermore, the fable-ist cannot even dispute the probabilistic inference concerning a theory's role in explaining success for otherwise they would allow for the possibility that the fable might one day become practical advice (given sufficient evidence). Rubinstein's \autocite*{rubinstein2012} canonical formulation of the theory-as-fable view subsumes the contributions of theory under those of ``basic intuitions and simulations'' (above). The goal of this section is to show why this move is wrongheaded.

The argument first proceeds by establishing the premises of the explanatory challenge version of the no-miracles argument. The successes have already been established ($P1$) so I first begin by isolating the common features of theory across these cases ($P4$). Second, multiple arguments are given for why the role of theory is not arbitrary with respect to explaining the successes of design economics ($P5$). In particular, the argument is constructed with reference to classic formulations of scientific realism in the context of the natural sciences as well as Rubinstein's own earlier work on interpretations of game theory. There turns out to be a surprising connection between the two, which, in my view, helps account for the role of theory in the successful instances of design economics. 

Though game theory is common to all instances of design economics, the extent to which it is used varies considerably. Notably, economist Tayfun Sönmez \autocite*{sönmez2023minimalist}, advocates for the use of an \textit{axiomatic methodology} which is derived from the work of Hervé Moulin \autocite*{moulin1988}. On his account, ``game theory, mechanism design, experimental, computational, and empirical techniques assum[e] supporting roles.'' \autocite[13]{sönmez2023minimalist}. Though it is fair to say that game theory is common to the successful instances of design economics (supporting premise $P4$), it is clear the degree to which game theory is used (let alone the particular theoretical model) varies considerably. Certainly, some theoretical primitives (e.g., random variables, utility functions, solution concepts, etc) are common throughout the range of successful cases; specific economic models and theories such as particular auction designs (English, Dutch, etc) are not. At a minimum, there is enough theoretical overlap between diverse fields within the broader approach of design economics to conclude that, for example, game theory is common auction design and the problem of residency matching. 

Another feature common to the multitude of theories behind the successful instances of design economics is that they model the choices the economist (designer) subsequently gets to take. As noted by philosopher of science Francesco Guala \autocite*[456, emphasis original]{guala2001}:
\begin{quote}
    [t]heory can be used to produce new technology, by shaping the social world so as to mirror a model in all its essential aspects. The `idealised' character of a theory may thus be a virtue rather than a defect, as the explicit role of theory is to point to a possibility. Theory \textit{projects}, rather than describing what is already there.
\end{quote}
\noindent The key idea here is that the world can be shaped to ``mirror a model'' (above). Guala calls this aspect of theory \textit{projective}. Theories of economic design can be formulated to reflect decisions economists get to take. Classic models like Myerson's \autocite*{myerson1981} optimal auction design model yield an auction format (an allocation and payment function) which can be subsequently implemented by the auction designer. This insight extends to counterexamples in theory that point to impossibilities in practice (for example, \cite{myerson1983}). Theoretical results reflect circumstances which might actually come to pass. And in coming to pass, these circumstances can be made to reflect the theory that brought them about. This fact is true in virtue of economists' ability to shape policy. This projective quality is common to the diverse economic models used across the range of successful cases that canvassed above. This is another common theoretical feature that these cases all share.

It is this ``mirroring'' that ultimately reduces the gap between reality and representation. This point was noted by Speigler \autocite*{spiegler2024} above, who remarked that it resulted in curbing that ``irony impulse''. Yet this mirroring can yield an incredibly tight correspondence between model and reality such as the social sciences have never before witnessed. For example, Google uses auction theory to determine which advertisements its users see. Google's chief economist Hal Varian \autocite*{varian2009} has even sketched the mathematical details of the problem that confronts Google. Notice; however, that any implementation will be entirely computational: the functional form of the auction used by Google approximates the real-valued functions used in the mathematical model arbitrarily well up to floating point error. This renders many of the models of design economics \textit{isomorphic} to their implementations in reality. Rubinstein \autocite*{rubinstein1991} explicitly denies this possibility\footnote{More specifically, in his opinion, this feature is undesirable: ``models are not supposed to be to isomorphic to reality'' \autocite[918]{rubinstein1991}. Here, he is clearly following Friedman's \autocite*{friedman1953} dictates on the role of unrealistic assumptions.}; however, I believe this rejection of my thesis is partly a function of it predating the success stories of the 1990s and the growth in electronic marketplaces and auctions, from Google and eBay to AirBnb and Amazon.

Could a fable-ist still press the issue? Could they grant that theory can be made to mirror the world---even approximating the real numbers up to floating point precision---but nonetheless fails to explain the successes of design economics? To do so, in light of the arguments presented above, they would need to maintain that the common theoretical elements of the successful cases are (1) reducible to ``basic intuitions'' \autocite[above]{rubinstein2012}\footnote{As noted above, accepting that theory explains the successes of design economics in conjunction with another factor (e.g., ``human simulations'' \autocite[above]{rubinstein2012}) still cedes too much ground: the resulting conclusion cannot be that economic theory is merely a collection of fables.} and (2) that insofar as they are isomorphic, they are isomorphic to the part of reality that does not matter.

With regards to (2) above, philosophers of science working in economics have previously noted that ``even if we grant that an auction model is partially isomorphic with reality, that fact is not very helpful in itself'' because we cannot know if the model ``is isomorphic to the part of reality that actually matters'' \autocite[309]{alexandrova2009}. Though this is undeniably true, the fact that the locus of the decision that a designer subsequently makes (the auction format, the matching algorithm, etc) can be modeled with the fidelity of the natural sciences---recall those earlier notions of literal, semantic or approximate truth as used in physics---represents a significant departure from a version of `local realism' for economics which rejects truth as a ``non-starter'' \autocite[above]{alexandrova2009}. I believe the reduction in the gap between reality and representation at the point of decision matters for bringing about the successes of design economics. This is certainly \textit{a} part that ``actually matters'' \autocite[above]{alexandrova2009}. It might not be the most significant feature in explaining success; it might be required in conjunction with many other features. However, the fact that (1) our theoretical models can be made to mirror the world and (2) that this mirroring is tight (up to isomorphism, in the case of electronic auctions) suggest that there are theoretical features of design economics that help explain its successes in ways that are non-arbitrary.

With regards to (1), the idea that the role of theory in explaining the successes of design economics is reducible to ``basic intuitions'', a number of economists working on auction design and matching problems strongly disagree. In the context of designing a matching algorithm for the NRMP (covered above), Al Roth goes as far as stating \autocite[1372]{roth2002}, ``[i]t turned out that the simple theory [of market design] offered a surprisingly good guide to the design, and approximated the properties of the large, complex markets fairly well''. On this account, theory is useful in the first sense dismissed by Rubinstein: it has genuine \textit{predictive value}. This strong view of the role of theory echoes the role of theory articulated by Milton Friedman \autocite*{friedman1953}. 

This view is not shared by all economists; however. In fact, the dominant view seems to be that ``the real value of the theory is in developing intuition'' \autocite[172]{mcafee1996}---the second view that Rubinstein \autocite*{rubinstein2012} rejects. Though individual contexts matter in determining which aspects of theory are relevant (i.e., there is no ``one size fits all'' approach \autocite[C94]{binmore2002}), the role of theory is to sharpen and guide the intuitions of the economist. This view is also advocated by Al Roth in conjunction with noted game theorist Robert Wilson \autocite*{roth2019}. It has even been dubbed the ``mainstream paradigm for market design'' by Tayfun Sönmez \autocite*[10]{sönmez2023minimalist}. Ultimately, this argument devolves into an attempt to reconcile mutually contradictory assertions between those who ``do not find any basis for claiming it was game theory that helped'' \autocite[above]{rubinstein2012} plan the successes of design economics and those who maintain the predictive value of theory or its role in shaping intuition \autocite{roth2019}. 

In my view, fable-ist attempts to cling to the position that theory does not explain the successes of design economics are increasingly hard to maintain in the face of the growing number and extent of the successes. I find the probabilistic nature of the no-miracles argument compelling. The alternative---that economic theory is a mere collection of fables and is not useful---not only runs headlong into the contravening claims of those very economists who brought about the successes of design economics but fails to account for the projective quality of these theories. To maintain theory is \textit{that} useless involves, effectively, invoking a miracle. And for this reason, I agree with the conclusion ($C2$) of the explanatory challenge version of the no-miracles argument: theory explains the success of design economics.


% This section offers a plausible account of theory (common feature $F$ of design economics) that explains the successes outlined in Section \ref{sec_design_econ} above. In isolating a feature common to all the successes of science, we render premise $P4$ probabilistically true: my argument hinges on making its rejection unlikely. To accomplish this goal we need both (1) a candidate feature $F$ and (2) an argument for that feature's role in explaining the success of a science. Following the work of Guala \autocite*{guala2001}, this section proposes that the \textit{projective} quality of the theory of design economics---the fact that designers can model choices they have control over and subsequently make---is unique\footnote{\color{red}TODO how to make this argument: (1) it characterizes all the successes; or (2) it is unique to design economics (ie not present anywhere else)? (1) is Mill's \textit{method of agreement} and aims to establish a necessary condition.} in the history of economics and explains the reason why theory contributed to the success of design economics. Thus, it would be unlikely if this common feature of design economics didn't help explain its success and, if this feature of the theory helps explain the success of design economics, the theory cannot amount to a collection of fables.

% Isolating a common theoretical feature of design economics is non-trivial. The body of theoretical work which economists-as-engineers (as in Roth \cite*{roth2002}) draw from covers the traditions of both matching and mechanism design. Moreover, much of the overlap is at the level of theoretical primitives: random variables, utility functions, etc. That this theory uses a particular mathematics, though true, would be an underwhelming conclusion. Furthermore, even within a particular theoretical tradition, there are differences in approach espoused by its leading practitioners. For example, within mechanism design, the well-known `Wilson Doctrine' amounts to the idea that ``practical mechanisms should be simple and designed without assuming that the designer has very precise knowledge about the economic environment in which the mechanism will operate'' \cite[\color{red}23]{milgrom2004}. Another well-known approach (called `Minimal Market Design') approaches questions of institutional design by focusing on ``(1) Legitimate [publicly stated] objectives of policymakers and other stakeholders in establishing the institution; (2) The existing structure of the institution'' \autocite[11]{sönmez2023minimalist}. Here, existing institutions and state objectives constrain the economist. These are just two examples of different ways economists tackle engineering problems and, therefore, entail different stipulations about the kind of theory that can be brought to bear\footnote{\color{red}TODO add commentary on counterexamples?}.

% The key common feature behind all theoretical approaches in design economics is that the economist (designer) is able to manipulate the environment they model. As noted by Guala:
% \begin{quote}
%     [t]heory can be used to produce new technology, by shaping the social world so as to mirror a model in all its essential aspects. The `idealised' character of a theory may thus be a virtue rather than a defect, as the explicit role of theory is to point to a possibility. Theory \textit{projects}, rather than describing what is already there. \autocite[456]{guala2001}
% \end{quote}
% \noindent The key idea here is that the world can be shaped to ``mirror a model'' (above). The sequencing is important: unlike a classical theoretical account of markets where a market is already assumed to exist and then its equilibrium is analyzed, the theories of design can be made to correspond arbitrarily closely to what they represent in virtue of the designer's ability to build the institution they model. Guala \autocite*{guala2001} is correct to note the analogy with minimizing traffic in public transport networks: these are artificial systems that are controlled and can be (re)engineered to attain a given objective.

% It is this ``mirroring'' that ultimately reduces the gap between reality and representation. This point was noted by Speigler \autocite*{spiegler2024} above, who remarked that it resulted in curbing that ``irony impulse''. Yet this mirroring can yield an incredibly tight correspondence between model and reality such as the social sciences have never before witnessed. For example, Google uses auction theory to determine which advertisements its users see. Hal Varian \autocite*{varian2009} has even sketched the mathematical details of the problem that confronts Google. Notice; however, that any implementation will be entirely computational: the functional form of the auction used by Google approximates the real-valued functions used in the mathematical model arbitrarily well \textit{up to floating point error}. The common projective quality of theories of design that explains the success of design economics can, I believe, be boiled down to the fact that the institution (or, borrowing again from Ross \cite*{ross2008}, ``structure'') is \textit{isomorphic}\footnote{\color{red}TODO see \autocite*{alexandrova2009} on isomorphism too.} to the model. Rubinstein \autocite*{rubinstein1991} explicitly denies this possibility\footnote{More specifically, in his opinion, this feature is undesirable: ``models are not supposed to be to isomorphic to reality'' \autocite[918]{rubinstein1991}.}; however, I believe this rejection of my thesis is partly a function of it predating the success stories of the 1990s and the growth in electronic marketplaces and auctions, from Google and Ebay to AirBnb and Amazon. 

% Contemporary digital markets are an excellent example of the shrinking gap between representation and reality but this does not account for the successes of the 1990s. Although software was essential in both cases of the FCC's auction design and the NRMP's residency matching; the idea that projective theory can be made isomorphic to reality extends to cover implementations that rely on actual laws or social norms for their rules. In my view, this mirroring quality of projective theory helps explain the success of design economics. Philosophers \autocite{alexandrova2009} and economists \autocite{roth2002} agree that both experimentation and computation are also essential for design economics. But there is no common theory nor theoretical approach across the success stories of design. There is; however, the projective quality of their theories. This quality fundamentally reduces the representational gap between a model and reality. In my view, it would miraculous if this didn't help explain the success of design economics. This is the minimal non-fable account of the role of economy theory in the domain of economic design.



\section{Conclusion}\label{fable_sec_conc}

The successes of design economics are hard to deny: from auction design to matching new entrants in labor markets, the corner of economics concerned with institutional design has demonstrated a remarkable ability to achieve a diverse range of objectives. Though the track record is far from spotless, successful auctions and matching markets have been replicated the world over, growing in complexity and scale. I have argued that these successes warrant an explanation. What features of the science of design economics can account for its successes? In doing so, I have argued for a `local realism' wherein we use the success of economics as a basis for grounding claims about economics' ability to interact with an external world.

Putting success at the center of a philosophical argument for (some form of) scientific realism straightforwardly recovers a no-miracles type argument for realism. However, instead of an argument for the truth of a scientific theory, here I focus on features of the science that contribute to explaining its success. And in the case of design economics, there are many such features. Those covered here include experiment and computation, as well as more ``political'' \autocite[1345]{roth2002} features such as personal and professional relationships. In my view, theory is also an important feature that explains the successes of design economics. This position has its critics. Notably, economic theorists like Ariel Rubinstein \autocite*{rubinstein2006,rubinstein2012} articulate something I call the \textit{theory-as-fable} view. On this view, theory is not to be taken seriously. It is not intended for prediction. It cannot offer policy recommendations. It does not even sharpen the intuitions of economists. Instead, economists should be ``satisfied even if the economic model is merely interesting'' \autocite[36]{rubinstein1982}. This deflationary view of economic theory removes it from contention as a potential feature that can explain the successes of designed economics.

Yet the fable-ist's position is not easy to maintain. First, there are clear theoretical commonalities in the range of successful instances of design economics. Despite differences in degrees of application, game theory is common to all of the successes of design economics. Furthermore, in my view, the \textit{projective} quality of these theories and the tightness between their representations and reality renders the explanation non-arbitrary (not reducible to random chance). The probabilistic inference of the explanatory challenge version of the no-miracles argument is then not so easily dismissed by a fable-ist. They cannot claim the conclusion `theory explains the successes of design economics' is merely unlikely (not miraculous) for then the fable might one day become practical advice. They cannot claim the conclusion is true in virtue of the fact that theory explains success only in conjunction with another feature. This would then make theory scientifically useful---a property fables do not possess! Thus, I do not think the falbe-ist's position is tenable. It cannot fit the facts: design economics has been successful and there exist theoretical commonalities which are not reducible to random chance. These commonalities help explain its success. To assert otherwise is to make a miracle of the contemporary science of economics.

I am far from alone in articulating this argument. Economists who have worked on the success stories of design economics espouse a range of alternative positions, from the view that ``the real value of the theory is in developing intuition'' \autocite[172]{mcafee1996}---again, this is the ``mainstream paradigm for market design'' \autocite[10]{sönmez2023minimalist}---to the stronger claim that that the theory of market design has predictive value \autocite{roth2002}. Rubinstein is nonetheless insistent: what matters, if theory matters at all, is ``basic intuitions'' \autocite[above]{rubinstein2012}. Perhaps there is a way to reconcile Rubinstein's views with those who claim the role of theory is to sharpen intuition. Rubinstein might simply underestimate what constitutes a ``basic'' intuition. In the words of economists Ken Binmore and Paul Klemperer \autocite*[C95]{binmore2002}, who consulted on the UK's 2000 3G spectrum auction, 
\begin{quote}
    [b]ut perhaps the most important lesson of all is not to sell ourselves too cheap. Ideas that seem obvious to a trained economist are often quite new to layfolk. 
\end{quote}
\noindent Future research could show a reconciliation is possible. What might be ``basic'' for Rubinstein might nonetheless represent a genuine scientific insight for those economists consult with. Ultimately, addressing the extent of this possibility is beyond the scope of this chapter.

In my view, economist Ran Spiegler \autocite*{spiegler2024} is correct to point out the subtle change in design economists' attitudes towards their work. He incisively writes, 
\begin{quote}
    [i]f I had to name one major shift in the sensibilities of economic theorists in the past half century, a prime candidate would be the way we conceptualize markets---from quasi-natural phenomena admired from afar to man-made institutions whose design can be tweaked by economist-engineers. \autocite[p137]{spiegler2024}
\end{quote}
\noindent Unlike Spiegler, I believe this change in perspective is unequivocally a good thing for \textit{science}. Markets, in a distinctly concrete sense, are socially constructed: we build them, regulate them, correct them, police them, etc. The change in sensibility brought about by design economics is mirrored in the projective quality of its theory. Economists model the control they exert over an environment or institution. This ``tweaking'' reflects a genuine rise in the power yielded by economists in the 21$^{\text{st}}$ Century. It remains to be seen whether this a good thing for \textit{society} \autocite{hitzigworking}.



