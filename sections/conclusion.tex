\chapter{Conclusion}

\section{Summary}

In chapter \ref{ch_fables} I propose a \textit{minimal non-fable} account of economic theory in the domain of design economics. This account of economic theory does not make the success of design economics a miracle. To argue for this account, I adapt a no-miracles argument for scientific realism to the scientific domain of economics. With this \textit{explanatory challenge} version of the no-miracles argument it becomes possible to philosophically interrogate the features of a successful science: are these features necessary to explain this science's successes? In the case of design economics, game theory is common to all its successes. Additionally, these theories are \textit{projective} \autocite{guala2001} in the sense that the world can be made to mirror the models developed by economists. Moreover, as in the case of electronic auctions, the gap between representation and reality can be made arbitrarily tight. This accounts for my rejection of Ariel Rubinstein's conviction that economic theorists are ``simply the tellers of fables'' \autocite[882]{rubinstein2006}.

Chapter \ref{ch_auctions} falls squarely in the tradition of design economics evaluated in the preceding chapter: the results are intended to ``suggest[] an agenda for future theoretical work'' \autocite[p1363]{roth2002}. I explore the properties of optimal multi-dimensional auctions in a setting where a single object of multiple qualities is sold to several buyers. Using simulations, I test the hypothesis that the optimal mechanism is an \textit{exclusive-buyer mechanism}, where buyers compete for the right to be the only buyer to choose between quality levels of a good. I find that in most multidimensional settings considered in this thesis, the exclusive-buyer mechanism well-approximates the revenue generated by the optimal mechanism and qualitatively matches the optimal (interim) allocations. However, the exclusive-buyer mechanism is clearly not optimal when the optimal mechanism yielded by simulations is not deterministic. Additionally, I provide evidence for the optimality of mechanisms with measure zero exclusion regions and find consistent evidence that the exclusion region remains constant with the number of buyers.

Finally, in chapter \ref{ch_reflexivity}, I offer a unified account of observer effects in the social sciences. I extend existing philosophical accounts of reflexivity to measurement, what I call the problem of \textit{reflexive measurement}. Additionally, my account draws from the extensive writings of social scientists who confront this measurement challenge as a practical matter. I argue that the problem of reflexive measurement is akin to the well-known idea of `measurement-as-intervention' in philosophy of economics \autocite{morgan2001}. I contend that what matters for reflexivity in science is whether agents are aware they are the subject of scientific investigation. In light of this understanding of reflexive measurement, I explore the extent to which mechanism design and, in particular, \textit{incentive compatible learning} can be used to mitigate the causal effects of social science on what it studies. I argue that the application of mechanism design to problems of measurement can be used to explicitly model the problem of misaligned incentives created when scientists investigate human behavior.




% \section{The Value of Incentive-Compatibility}

% \begin{itemize}
% 	\item \autocite{hitzig2024} + Rawls
% 	\item \autocite{wollesen2024, luce95}
% \end{itemize}



\section{Future Work}

The conclusions concerning optimal multidimensional auctions in chapter \ref{ch_auctions} invite further study of the exclusive-buyer mechanism. This mechanism is clearly optimal in cases where randomization is not required (conjecture \ref{conj_rev}) and captures the qualitative behavior of the allocations (conjecture \ref{conj_alloc}). Furthermore, it is possible to extend the exclusive-buyer mechanism to include stochastic contracts, which entails that a more general formulation of this kind of mechanism may be optimal for the general multidimensional setting of a single good with multiple quality levels. The assumptions of identical bidder valuations and linear virtual values may drive the simulation results and alternative specifications should be explored. The exclusive-buyer mechanism is also compatible with measure zero exclusion regions in multidimensional settings, an unexpected phenomenon that merits further exploration.

The combined conclusions of chapters \ref{ch_fables} and \ref{ch_reflexivity} point to the concrete application of economic theory to problems of measurement. Recent theoretical research on incentive-compatible estimators \autocite{caragiannis2016} and data acquisition procedures \autocite{bates2022, roth2012surveys} has, to the best of my knowledge, yet to be applied in practice. Insofar as one can argue for the value of economic theory in bringing about the successes of design economics, this theory should be put to use in designing better measurements. To what extent `designing' a measurement procedure is similar to the problems of design economics is an open question. In my view, answering this question is of paramount importance for social scientists seeking a  ``more advanced theoretical knowledge of our measuring instruments’’ \autocite[1231]{achen1975}.